%
% Template for making SDDS Toolkit manual entries.
%
\begin{latexonly}
\newpage
\end{latexonly}

%
% Substitute the program name for sddssynchlog
%
\subsection{sddssynchlog}
\label{sddssynchlog}

\begin{itemize}
\item {\bf description:}
%
% Insert text of description (typicall a paragraph) here.
%
\verb+sddssynchlog+ Reads values of process variables synchronously
and writes them to an output file.  Synchronism is imposed by
requiring that all PVs generated callbacks occur within the number of
seconds given by the -synchInterval argument.
\item {\bf example 1:} 
%
% Insert text of examples in this section.  Examples should be simple and
% should be proceeded by a brief description.  Wrap the commands for each
% example in the following construct:
% 
%

The linac bpms and MV200 PVs are synchonously monitored using the
command below.  One might wish to monitor these PVs determine if linac
beam position and transverse size at a flag are correlated.  The
output file will contain 100 aligned samples of the PV data and should
therefore take 16.67 seconds assuming the linac is running at 6 Hz.
Alignment of process variable data is accepted and written to the
output file if and only if the PV time stamps are withing +/- 1/6 Hz =
0.08 seconds.  Data acquisition stops after 100 samples are acquired
or a -timelimit of 20 seconds has elapsed.  Note, it is absolutely
required that the scalar synchronous PV file listed below
(synchPVs.mon) contain at least a single PV.  This is true even if one
wishes only to synchronously acquire waveform process variables.
\begin{verbatim}
sddssynchlog synchPVs.mon synchPVs.sdds -samples=100 -synchInterval=0.08 
-timelimit=20,seconds -verbose
\end{verbatim}
where the contents of the file \verb+synchPVs.mon+ are
\begin{verbatim}
SDDS1
&description &end
&column
 name = ControlName,  type = string, &end
&column
 name = ControlType,  type = string, &end
&column
 name = ReadbackUnits,  type = string, &end
&column
 name = ReadbackName,  type = string, &end
&data
 mode = ascii, no_row_counts=1 &end
! page number 1
L2:PM1:BPM.CX  pv mm L2:PM1:BPM.XPOS 
L2:PM1:BPM.CY  pv mm L2:PM1:BPM.YPOS 
L2:PM2:BPM.CX  pv mm L2:PM2:BPM.XPOS 
L2:PM2:BPM.CY  pv mm L2:PM2:BPM.YPOS 
LI:VD1:x:raw:cal:sigmaM pv Pixels LVid:xRawSigma 
LI:VD1:y:raw:cal:sigmaM pv Pixels LVid:yRawSigma 
...

\end{verbatim}
\item {\bf example 2:} 
%
% Insert text of examples in this section.  Examples should be simple and
% should be proceeded by a brief description.  Wrap the commands for each
% example in the following construct:
% 
%

The scalar linac bpm and MV200 PVs are synchonously monitored along
with LCLS bpm data waveforms using the command below.  The command is
identical to that in example 1 except for the additional waveform file
command which that specifies a file containing waveform PV data.
\begin{verbatim}
sddssynchlog synchPVs.mon synchPVs.sdds -waveformData=LCLS.wmon
-samples=100 -synchInterval=0.08 -timelimit=20,seconds -verbose
\end{verbatim}
where the contents of the file \verb+LCLS.wmon+ are
\begin{verbatim}
SDDS1
&parameter name=WaveformLength, type=long, &end
&column
 name=WaveformPV, type=string,  &end
&column
 name=WaveformName, type=string,  &end
&data mode=ascii, &end
! page number 1
128
                   9
PAD:K211:1:CH0_RAW_WF PAD:K211:1:CH0_RAW_WF 
PAD:K211:1:CH1_RAW_WF PAD:K211:1:CH1_RAW_WF 
PAD:K211:1:CH2_RAW_WF PAD:K211:1:CH2_RAW_WF 
PAD:K211:2:CH0_RAW_WF PAD:K211:2:CH0_RAW_WF 
PAD:K211:2:CH1_RAW_WF PAD:K211:2:CH1_RAW_WF 
PAD:K211:2:CH2_RAW_WF PAD:K211:2:CH2_RAW_WF 
PAD:K211:3:CH0_RAW_WF PAD:K211:3:CH0_RAW_WF 
PAD:K211:3:CH1_RAW_WF PAD:K211:3:CH1_RAW_WF 
PAD:K211:3:CH2_RAW_WF PAD:K211:3:CH2_RAW_WF 
...
\end{verbatim}
\item {\bf synopsis:} 
%
% Insert usage message here:
%
\begin{verbatim}
sddssynchlog <input> <output> [-slowData=<input>]
[-waveformData=<filename>] -samples=<number> [-timeLimit=<value>[,<units>]]
-eraseFile [-pendIOTime=<seconds>] [-connectTimeout=<seconds>]
[-verbose] [-comment=<parameterName>,<text>]
[-synchInterval=<seconds>] [-precision={single|double}]
[-saveTimeStamps=<filename>] [-steps=<int>] [-interval=<value>[,<units>]] 

Logs numerical data for the process variables named in the ControlName
column of <input> to an SDDS file <output>.  Synchronism is imposed by
requiring that all PVs generated callbacks within the number of seconds
given by the -synchInterval argument.
\end{verbatim}
\item {\bf files:}
% Describe the files that are used and produced
\begin{itemize}
\item {\bf input file:}\par
The input file is an SDDS file with a few data columns required:
\begin{itemize}
        \item {\tt ControlName} or {\tt Device} --- Required string column for the names of the process variables
                or devices to be monitored. Both column names are equivalent.
        \item {\tt Message} --- Optional string column for the device read message. If a row entry in
                column {\tt ControlName} is a process variable, then the corresponding entry
                in {\tt Message} should be a null string. 
        \item {\tt ReadbackName} --- Optional string column for the names of the data columns in the 
                output file. If absent, process variable or device name is used.
        \item {\tt ReadbackUnits} --- Optional string column for the units fields of the data columns in the 
                output file.If absent, units are null.
        \item {\tt ScaleFactor} --- Optional double column for a factor with which to multiply
                values of the readback in the output file.
\end{itemize}

\item {\bf waveformData file:} \par The waveformData file is an
optional input file specified on the command line which lists waveform
PVs to be synchronously logged along with the scalar PVs in the input
file.
\begin{itemize}
        \item {\tt WaveformPV} --- Required string column containing the names of the waveform
                process variables to be monitored.
        \item {\tt WaveformName} --- Optional string column for the names of the waveform data
                columns in the output file.
        \item {\tt WaveformLength} --- Required long parameter giving the number of waveform elements 
                to be acquired.  It can be less than the maximum number of waveform elements.
\end{itemize}

\item {\bf slowData file:} \par The slowData file is an optional input
file specified on the command line which lists scalar PVs to be slow
(read non-synchronously) logged along with the synchronously logged
scalar or waveform PVs in the input and waveform PV files.  This feature is
convenient if one wishes to log PVs for system information purposes instead of
correlation analysis (such as beam current, vacuum pressure etc.).  This file has
the same columns as the input file.

\item {\bf output file:}\par The output file contains one data column
for each process variables named in the input file if no waveformData
file is specified. By default, the data type is float (single
precision).  If an optional waveformData file is specified, the
structure of the output file changes.  In this case, the column data
are the waveform PVs and the scalar PVs in the input and slowData
files are parameters.  Each page contains waveform and scalar PV data
for a single sample of the total number of samples acquired.  The
total number of pages equal the number of samples specified in
-samples if all samples are able to be collected (ie. timeLimit is not
exceeded during acquisition).  Time columns and other miscellaneous
columns are defined:
\begin{itemize}
        \item {\tt Time} --- Double column of time since start of epoch. This time data can be used by
        the plotting program {\verb+sddsplot+} to make the best coice of time unit conversions
        for time axis labeling.
        \item {\tt Sample} --- Long column for sample number.
\end{itemize}
Many time-related parameters are defined in the output file:
\begin{itemize}
        \item {\tt TimeStamp} --- String parameter for time stamp for file.
        \item {\tt PageTimeStamp} --- String parameter for time stamp for each page. When data
                is appended to an existing file, the new data is written to a new
                page. The {\tt PageTimeStamp} value for the new page is the creation
                date of the new page. The {\tt TimeStamp} value for the new page is the creation 
                date of the very first page.
        \item {\tt StartTime} --- Double parameter for start time from the {\tt C} time call cast to type double.
        \item {\tt YearStartTime} --- Double parameter for start time of present year from the {\tt C} time call cast to type double.
        \item {\verb+StartYear+} --- Short parameter for the year when the file was started.
        \item {\verb+StartJulianDay+} --- Short parameter for the day when the file was started.
        \item {\verb+StartMonth+} --- Short parameter for the month when the file was started.
        \item {\verb+StartDayOfMonth+} --- Short parameter for the day of month when the file was started.
        \item {\verb+StartHour+} --- Short parameter for the hour when the file was started.
\end{itemize}
\end{itemize}
%
\item {\bf switches:}
%
% Describe the switches that are available
%
    \begin{itemize}
%
        \item {\tt -samples} --- Specifies the number of samples to attempt.  If synchronism problems
                are encountered, fewer samples will appear in the output.
        \item {\tt -timeLimit} --- Specifies maximum time to wait for data collection.
        \item {\tt -eraseFile} --- Specifies erasing the file <output> if it exists already.
        \item {\tt -pendIOtime=<value>} --- Obsolete.
        \item {\tt -connectTimeout=<value>} --- Specifies maximum time in seconds to wait for a connection before
                issuing an error message. 60 is the default.
        \item {\tt -verbose} --- Specifies printing of possibly useful data to the standard output.
        \item {\verb+-comment=<parameterName>,<text>+} ---
                Gives the parameter name for a comment to be placed in the SDDS output file,
                along with the text to be placed in the file.
        \item {\tt -synchInterval} --- Specifies the time spread allowed for callbacks from the PVs being logged.
                If any PV fails to callback within the specified interval, the data for
                that sample is discarded.
        \item {\tt -slowData} --- Specifies an sddsmonitor-type input file giving the names of PVs to be
                acquired at a "slow" rate, i.e., without synchronization.  Data for these PVs
                are interpolated linearly to get values at each time sample of the synchronous
                data.
        \item {\tt -waveformData} --- Specifies an sddswmonitor-type input file giving the names of waveform PVs to
                be synchronously logged.  If this option is given, the output file is changed
                so that waveform data is stored in columns while scalar data is stored in
                parameters.
        \item {\tt -saveTimeStamps} --- Specifies the name of a file to which to write raw time-stamp data for the
                synchronously-acquired channels.  This can be useful in diagnosing the source
                of poor sample alignment.
        \item {\tt -precision} Specifies PV data to be either single or double precision.  Default is double precision.  
        \item {\tt -steps} specifies how many synchlogs will be taken with -samples of data, default is 1.
        \item {\tt -interval} specifies the waiting time between two synchlog steps.
    \end{itemize}

\item {\bf see also:}
    \begin{itemize}
%
% Insert references to other programs by duplicating this line and 
% replacing <prog> with the program to be referenced:
%
    \item \progref{sddsmonitor}
    \item \progref{sddswmonitor}
    \item \progref{sddsvmonitor}
    \end{itemize}
%
% Insert your name and affiliation after the '}'
%
\item {\bf author: M. Borland, ANL} 
\end{itemize}
