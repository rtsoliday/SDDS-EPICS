% sddscaramp: ramp process variables between current and target states.
\begin{sddsprog}{sddscaramp}
\item {\bf description:}
\verb+sddscaramp+ performs ramping of process variables between
the present state and the states in one or more SDDS files.

\item {\bf example:} 
% 
The following example shows how one would use \verb+sddscaramp+ to
ramp 50\% of the way to a new steering configuration, using 10
steps and pausing 1 second between steps: 
\begin{verbatim}
sddscaramp -rampTo=steering.snap,steps=10,pause=1,percentage=50
\end{verbatim}
The file \verb+steering.snap+ contains corrector magnet power supply
settings, such as might be saved to an SDDS file using the
\verb+sddssnapshot+ program.
\item {\bf synopsis:} 
\begin{verbatim}
sddscaramp
  -rampTo=<filename>,steps=<number>,pause=<seconds>[,percentage=<value>][,differential=<factor>]
  [-rampTo ...] [-controlNameColumn=<string>] [-dataColumn=<columnName>]
  [-pendIOtime=<seconds>] [-resendFinal] [-verbose]
\end{verbatim}
\item {\bf files:}

The input files are SDDS files.  There must be a string column named
{\tt ControlName}, {\tt Device}, or {\tt DeviceName} that contains the
process variable names.  There must also be a string column named {\tt
ValueString}, a numerical column named {\tt Value}, or a column of
either type with the name specified by the \verb+-dataColumn+ option;
this column contains the final value for the corresponding process
variable.  

When data is supplied in a string column, {\tt sddscaramp} needs a way
to determine if the data value is actually a number rather than a
literal string value (e.g., an enumerated value).  The optional {\tt
IsNumerical} column can be used for this purpose.  If supplied, this
column should contain character values {\tt y} or {\tt n}, indicating
that each PV (respectively) does or does not have numerical values.
If the {\tt IsNumerical} column does not exist or is not of character
type, then {\tt sddscaramp} uses an internal algorithm to decide
whether the data for each PV is numerical or not.  This may fail in
the case of enumerated values that contain numbers, resulting in
incorrectly restored values.  For reliable results, the use of {\tt
IsNumerical} with string data is required.  If the data is in a
numerical column to begin with, of course, there is no ambiguity.

\item {\bf switches:}
    \begin{itemize}
        \item {\tt -rampTo={\em filename},steps={\em
        number},pause={\em seconds},[percentage={\em value}][,differential={\em factor}]} ---
        Specifies a file to ramp to, the number of steps in the ramp,
        and the time to wait between sending setpoints for each step.
        Optionally, one may specify ramping only part of the way to
        the configuration in the file or offsetting by a scaled
        amount using the {\tt differential} factor.  If several
        {\tt -rampTo} options are given, {\tt sddscaramp} ramps to each
        of them in the order given.
        \item {\tt -controlNameColumn={\em name}} --- Specifies the
        name of the column containing the PV names when it is not
        one of {\tt ControlName}, {\tt Device}, or {\tt DeviceName}.
        \item {\tt -dataColumn={\em name}} --- Specifies the name of
        the column in the input files that contains the PV values.
        By default, the program uses {\tt ValueString} or {\tt Value}.
        \item {\tt -pendIOtime={\em seconds}} --- Sets the time to
        wait for channel access operations to complete.
        \item {\tt -resendFinal} --- Resends the final setpoint for
        numerical PVs after ramping is complete.
        \item {\tt -verbose} --- If given, informational text is
        printed out as the ramping proceeds.
      \end{itemize}

\item {\bf see also:}
    \begin{itemize}
% replacing <prog> with the program to be referenced:
    \item \progref{sddssnapshot}
    \item \progref{sddscasr}
    \end{itemize}
\item {\bf author: M. Borland (ANL)}
\item {\bf programmers: R. Soliday, M. Borland (ANL)}
\end{sddsprog}
