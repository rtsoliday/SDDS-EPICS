% sddslogger: log PV values at intervals defined in an input file.
\begin{sddsprog}{sddslogger}
\item \textbf{description:}
\verb+sddslogger+ reads values of process variables and writes them to a file at a specified time interval.
One or more input files defines the process variables to be monitored.
\item \textbf{examples:}
%
The pressure readbacks of storage ring ion pumps and the stored current are monitored
with the command below.
\begin{verbatim}
sddslogger SRvac.mon SRvac.sdds -time=24,hours -sampleInterval=1,minutes
\end{verbatim}
where the contents of the file \verb+SRvac.mon+ are
\begin{verbatim}
SDDS1
&description &end
&column
 name = ControlName,  type = string, &end
&column
 name = ControlType,  type = string, &end
&column
 name = ReadbackUnits,  type = string, &end
&column
 name = ReadbackName,  type = string, &end
&data
 mode = ascii, no_row_counts=1 &end
! page number 1
S35DCCT:currentCC pv mA S35DCCT
VM:01:3IP1.VAL pv Torr VM:01:3IP1
VM:01:2IP2.VAL pv Torr VM:01:2IP2
VM:01:2IP3.VAL pv Torr VM:01:2IP3
...
\end{verbatim}
\item \textbf{synopsis:}
\begin{verbatim}
usage: sddslogger <SDDSinputfile1> <SDDSoutputfile1> <SDDSinputfile2> <SDDSoutputfile2>...
    [-generations[=digits=<integer>][,delimiter=<string>][,rowlimit=<number>][,timelimit=<secs>] | -dailyFiles | -monthlyFiles]
    [-sampleInterval=<real-value>[,<time-units>]]
    [-logInterval=<integer-value>] [-flushInterval=<integer-value>]
    [-steps=<integer-value> | -time=<real-value>[,<time-units>]]
    [-enforceTimeLimit] [-offsetTimeOfDay]
    [-verbose] [-singleshot{=noprompt | stdout}]
    [-precision={single|double}]
    -onerror={usezero|skiprow|exit} [-pendIOtime=<value>]
    [-conditions=<filename>,{allMustPass | oneMustPass}[,touchOutput][,retakeStep]]
Writes values of process variables or devices to a binary SDDS file.
\end{verbatim}
\item \textbf{files:}
\begin{itemize}
  \item \textbf{input file(s):}\par
      The input files are SDDS files with a few data columns required:
  \begin{itemize}
    \item {\tt ControlName} or {\tt Device} --- Required string column for the names of the process variables
          or devices to be monitored. Both column names are equivalent.
    \item {\tt ReadbackUnits} --- Required string column for the units fields of the data columns in the
          output file.
    \item {\tt ReadbackName} --- Optional string column for the names of the data columns in the
          output file. If absent, process variable or device name is used.
    \item {\tt Message} --- Optional string column for the device read message. If a row entry in
          column {\tt ControlName} is a process variable, then the corresponding entry
          in {\tt Message} should be a null string.
    \item {\tt ScaleFactor} --- Optional double column for a factor with which to multiply
          values of the readback in the output file.
    \item {\tt Average} --- Optional long column.  If value is non-zero the process variable will have
          its average value logged.  Otherwise it will log the most recent value.
          values of the readback in the output file.
    \item {\tt DoublePrecision} --- Optional long column. If value is non-zero the process variable
          will be logged as a double-precision number.  Otherwise it will be logged as a single-precision
          number.
  \end{itemize}

  \item \textbf{conditions file:} \par
      The conditions file is an optional input file specified on the command line which lists
      conditions that must be satisfied at each time step before the data can be logged.

      The file is like the main input file, but has numerical columns \verb+LowerLimit+ and \verb+UpperLimit+.
      The minimal column set is \verb+ControlName+, which contain the PV names, and the two limits columns above.
      Depending on command line options, when any or all PV readback from this file
      is outstide the range defined by the corresponding data from \verb+LowerLimit+ and \verb+UpperLimit+,
      none of the data of the PVs in the input files are recorded.
      When this situation occurs for a long period of time, the size of the output files doesn't
      change, and it may appear that the monitoring process has somehow stopped.
      It is possible to check the program activity with the \verb+touch+ sub-option
      which causes the logging program to touch the output file at every step.

  \item \textbf{output file(s):}\par
      The output files contains one data column for each process variable named in the corresponding input file. By default,
      the data type is float (single precision).
      Time columns and other miscellaneous columns are defined:
  \begin{itemize}
    \item {\tt Time} --- Double column of time since start of epoch. This time data can be used by
          the plotting program {\verb+sddsplot+} to make the best choice of time unit conversions
          for time axis labeling.
    \item {\tt TimeOfDay} --- Float column of system time in units of hours.
          The time does not wrap around at 24 hours.
    \item {\tt DayOfMonth} --- Float column of system time in units of days.
          The day does not wrap around at the month boundary.
    \item {\tt Step} --- Long column for step number.
    \item {\tt CAerrors} --- Long column for number of channel access errors at each reading step.
  \end{itemize}

      Many time-related parameters are defined in the output file:
  \begin{itemize}
    \item {\tt TimeStamp} --- String parameter for time stamp for file.
    \item {\tt PageTimeStamp} --- String parameter for time stamp for each page. When data
          is appended to an existing file, the new data is written to a new
          page. The {\tt PageTimeStamp} value for the new page is the creation
          date of the new page. The {\tt TimeStamp} value for the new page is the creation
          date of the very first page.
    \item {\tt StartTime} --- Double parameter for start time from the {\tt C} time call cast to type double.
    \item {\tt YearStartTime} --- Double parameter for start time of present year from the {\tt C} time call cast to type double.
    \item {\verb+StartYear+} --- Short parameter for the year when the file was started.
    \item {\verb+StartJulianDay+} --- Short parameter for the day when the file was started.
    \item {\verb+StartMonth+} --- Short parameter for the month when the file was started.
    \item {\verb+StartDayOfMonth+} --- Short parameter for the day of month when the file was started.
    \item {\verb+StartHour+} --- Short parameter for the hour when the file was started.
  \end{itemize}
\end{itemize}

\item \textbf{switches:}
\begin{itemize}
%   \item {\tt -pipe[=input][,output]} --- The standard SDDS Toolkit pipe option.
  \item {\verb+-generations[=digits=<integer>][,delimiter=<string>]+} ---
      The output is sent to the file \verb+<SDDSoutputfile>-<N>+, where \verb+<N>+ is
      the smallest positive integer such that the file does not already
      exist.  By default, four digits are used for formatting \verb+<N>+, so that
      the first generation number is 0001.
  \item {\tt -dailyFiles} --- Starts a new output file each day. Optional qualifiers
      \verb+rowlimit+, \verb+timelimit+, \verb+timetage+, and \verb+verbose+ control
      file rollover behavior.
  \item {\tt -monthlyFiles} --- Starts a new output file each month with the same optional
      qualifiers as \verb+-dailyFiles+.
  \item {\tt -sampleInterval=<real-value>[,<time-units>]} --- Specifies the interval between readings. The time
      interval is implemented with a call to usleep between calls to the control system.
      Because the calls to the control system may take up a significant amount of time, the average
      effective time interval may be longer than specified.
  \item {\tt -logInterval=<interval>} --- Specifies the number of sampling intervals to average before
      writing to the output file.
  \item {\tt -flushInterval=<interval>} --- Number of sampling intervals between forced flushes of data to disk.
  \item {\tt -steps=<integer-value>} --- Number of readbacks for each process variable before normal exiting.
  \item {\tt -time=<real-value>[,<time-units>]} --- Total time for monitoring. Valid time units are
      seconds, minutes, hours, and days. The program calculates the number of steps by dividing this time
      by the interval. The completion time may be longer, because the time interval in not guaranteed.
  \item {\tt -enforceTimeLimit} --- Enforces the time limit given even if the expected number of samples has
      not been taken.
  \item {\tt -offsetTimeOfDay} --- Adjusts the starting TimeOfDay value so that it corresponds to the
      day for which the bulk of the data is taken.  Hence, a 26 hour job started at 11pm would have an
      initial time of day of -1 hour and a final time of day of 25 hours.
  \item {\tt -verbose} --- Prints out a message when data is taken.
  \item {\verb+-singleshot[=noprompt]+} --- a single read is prompted at the terminal
      and initiated by a \verb+<cr>+ key press. The time interval is disabled.
      With \verb+noprompt+ present, no prompt is written to the terminal, but a \verb+<cr>+
      is still expected. Typing ``q'' or ``Q'' terminates the monitoring.
  \item {\tt -precision={single|double}} --- Sets floating-point precision for logged data.
  \item {\tt -onerror=\{usezero|skiprow|exit\}} --- Selects action taken when a channel access error occurs.
      The default is using zero ({\tt usezero}) for the value of the process variable
      with the channel access error, and resuming execution. The second option ({\tt skiprow}) is to
      simply throw away all the data for that read step, and resume execution.
      the third option is to exit the program.
  \item {\tt -pendIOtime=<value>} --- Sets the maximum time to wait for connection to each PV.
  \item {\verb+-conditions=<filename>,{allMustPass | oneMustPass}[,touchOutput][,retakeStep]+} ---
      Names an SDDS file containing PVs to read and limits on each PV that must
      be satisfied for data to be taken and logged.  The file is like the main
      input file, but has numerical columns LowerLimit and UpperLimit.

      One of \verb+allMustPass+ or \verb+oneMustPass+ must be specified. It would make sense
      to use \verb+allMustPass+ in most monitoring applications.
      If \verb+touchOutput+ is present, then the output file is touched, even if no data
      is written. This way, one can determine by the time stamp of the file
      whether the monitoring job is still alive
      when the conditions fail for a long period of time. If \verb+retakeStep+ is
      present, then the value of \verb+Step+ in the output file is not
      incremented until the conditions pass, and data is written to the output file.
\end{itemize}

\item \textbf{see also:}
\begin{itemize}
% replacing <prog> with the program to be referenced:
  \item \progref{sddsmonitor}
  \item \progref{sddsvmonitor}
  \item \progref{sddswmonitor}
  \item \progref{sddssnapshot}
\end{itemize}
\item \textbf{author:} R. Soliday, M. Borland, H. Shang ANL
\end{sddsprog}
