\begin{sddsprog}{sddspvasaverestore}
\item \textbf{description:}
  \verb+sddspvasaverestore+ saves or restores EPICS process variable values using the EPICS PVA
  interface. It reads an SDDS input file listing process variables and either writes current
  values to an output file or sets the PVs to values from the input file. Options allow daemon
  operation, run control integration, trigger PVs, and waveform handling.
\item \textbf{examples:}
\begin{flushleft}{\tt
sddspvasaverestore pvlist.sdds snapshots/sr -save -dailyFiles -logFile=save.log\\
}\end{flushleft}
\item \textbf{synopsis:}
\begin{flushleft}{\tt
sddspvasaverestore <inputfile> <outputRoot> [-pipe=[input|output]]\
  [-save] [-restore[=verify]]\
  [-daemon]\
    [-inputFilePV=<pvname>,<provider>] [-outputFilePV=<pvname>,<provider>]\
    [-triggerPV=<string>,<provider>]\
    [-description=<string>|pv=<pvname>,provider=<provider>]\
  [-interval=<seconds>] [-pendIOTime=<seconds>]\
  [-dailyFiles] [-semaphore=<filename>] [-logFile=<filename>]\
  [-unique] [-pidFile=<pidFile>] [-verbose]\
  [-waveform=[rootname=<string>][,directory=<string>][,extension=<string>][,pingTimeout=<seconds>][,onefile][,savewaveformFile][,fullname]]\
  [-numerical]\
  [-runControlPV={string=<string>|parameter=<string>},pingTimeout=<value>,pingInterval=<value>]\
  [-runControlDescription={string=<string>|parameter=<string>}]
}\end{flushleft}
\item \textbf{files:}
\begin{itemize}
  \item \textbf{input file:} SDDS file listing PVs to save or restore, including columns \verb|ControlName| and \verb|ValueString| for restoring.
  \item \textbf{output file:} SDDS file receiving saved PV values when \texttt{-save} is used.
\end{itemize}
\item \textbf{switches:}
\begin{itemize}
  \item {\tt -save} --- read PV values and write them to the output file.
  \item {\tt -restore[=verify]} --- write values from the input file to PVs; optional {\tt verify} reads back each value.
  \item {\tt -daemon} --- run in background, waiting for file changes or trigger PV activity.
  \item {\tt -dailyFiles} --- append the current date to the output filename.
  \item {\tt -interval=<seconds>} --- loop interval when in daemon mode or when checking a trigger PV.
  \item {\tt -triggerPV=<pv>,<provider>} --- PV that causes a save when it changes from 0 to 1.
  \item {\tt -inputFilePV=<pv>,<provider>} --- PV providing the input filename.
  \item {\tt -outputFilePV=<pv>,<provider>} --- PV receiving the output filename.
  \item {\tt -description=<string>|pv=<pv>,provider=<provider>} --- snapshot description or PV supplying it.
  \item {\tt -waveform=[...]} --- configure saving or restoring waveform PV data.
  \item {\tt -runControlPV={string=<string>|parameter=<string>},pingTimeout=<value>,pingInterval=<value>} --- run control record and ping settings.
  \item {\tt -runControlDescription={string=<string>|parameter=<string>}} --- description string for run control logging.
  \item {\tt -numerical} --- output numeric strings for enumerated PV types.
  \item {\tt -pendIOTime=<seconds>} --- maximum time to wait for connections and value returns.
  \item {\tt -semaphore=<filename>} --- flag file written when connections are complete.
  \item {\tt -pidFile=<file>} --- file to store process ID.
  \item {\tt -logFile=<file>} --- log file for program messages.
  \item {\tt -unique} --- remove duplicate PVs from the input file.
  \item {\tt -verbose} --- print progress messages.
  \item {\tt -pipe[=input][,output]} --- standard SDDS Toolkit pipe option.
\end{itemize}
\item \textbf{see also:}
\begin{itemize}
  \item \progref{sddssnapshot}
\end{itemize}
\item \textbf{author:} H. Shang and R. Soliday, ANL
\end{sddsprog}
