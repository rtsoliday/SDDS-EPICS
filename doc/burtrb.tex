%
% Template for making SDDS Toolkit manual entries.
%
\begin{latexonly}
\newpage
\end{latexonly}

%
% Substitute the program name for burtrb
%
\subsection{burtrb}
\label{burtrb}

\begin{itemize}
\item {\bf description:}
%
% Insert text of description (typically a paragraph) here.
%
\verb+burtrb+ reads values of process variables and writes them to a file.
An input file lists the process variables to be read.
The output file (also called a snapshot file) can be used by program \verb+burtwb+
to restore the process variables.
\item {\bf example:} 
%
% Insert text of examples in this section.  Examples should be simple and
% should be proceeded by a brief description.  Wrap the commands for each
% example in the following construct:
% 
%
The state of the APS storage ring is saved in the snapshot file SR.snp using the 
request file SR.req:
\begin{verbatim}
burtrb -f SR.req -o SR.snp
\end{verbatim}
where the contents of the file \verb+SR.req+ are
\begin{verbatim}
SDDS1
&column
 name = ControlType,  type = string, &end
&column
 name = ControlName,  type = string, &end
&data
 mode = "ascii", no_row_counts=1 &end
pv      S1A:Q1:CurrentAO        
pv      S1A:Q2:CurrentAO
...
\end{verbatim}
Note that the header contain the minimal amount of information. There may be situations
where more columns need to be defined, as described elswhere in this manual.
\item {\bf synopsis:} 
%
% Insert usage message here:
%
\begin{verbatim}
usage: burtrb -f req1 {req2 ...} {-l logfile} {-o outfile} {-d} {-v}
	{-c ... comments ...} {-k keyword1 ... keywordn}
	{-r retry count} {-sdds or -nosdds} {-Dname{=def}}
	{-Ipathname}

where

  -f req1 {req2 ...} - Request filenames. This is the only switch
  	that is not optional.  You must specify at least one request
  	file.

  -l logfile - Log filename. The name of the file where all logging
  	messages (e.g. error messages, reports of process variables
  	that were not found) go.  The default is stderr.

  -o outfile - Snapshot filename.  The name of the file where the
  	snapshot information goes.  The default is stdout.

  -d - Debug.  Save the files created by processing the request
  	files with the C preprocessor.  The default is to delete these
  	files.

  -v - Verbose.  This increases the amount of information displayed
  	in the logfile.

  -c ... comments ... - Comments.  Adds comments to the header of
  	the snapshot file.

  -k keyword1 ... keywordn - Keywords.  Adds keywords to the header
  	of the snapshot file.

  -r retry count - Number of additional attempts to wait for
  	connections.  The program will attempt to find all the process
  	variables.  If it is unsuccessful, it will try this many more
  	times to establish connections.  The default value is 0.

  -sdds or -nosdds - SDDS/non-SDDS snapshot file.  Explicitly
  	specifying that the generated snapshot file will be 
  	SDDS/non-SDDS compliant.  The default is to adopt the SDDS
  	type from the input(s).  If there is a heterogeneous set of
  	inputs (some SDDS and some non-SDDS), the default is to produce
  	and SDDS compliant snapshot file.

\end{verbatim}
\item {\bf files:}
% Describe the files that are used and produced
\begin{itemize}
\item {\bf input file:} \par
The input file is an SDDS file with at least two columns:
\begin{itemize}
        \item {\tt ControlName} --- Required string column for the process variable or device names.
        \item {\tt ControlType} --- Required string column for the control name type. For a 
                process variable name use ``pv''; for a device name use ``dev''.
\end{itemize}
The optional columns are:
\begin{itemize}
        \item {\tt BackupMsg} --- String column for the device read message if the {\tt ControlType} value is ``dev''. 
                This column is transferred to the output file. 
                If this column is absent then the default read message is ``read'', and the column {\tt BackupMsg} is created in the output file.
        \item {\tt RestoreMsg} --- String column for the device set message if the {\tt ControlType} value is ``dev''.
                This column is transferred to the output file. 
                If this column is absent then the default read message is ``read'', and the column {\tt RestoreMsg} is created in the output file.
\end{itemize}

\item {\bf output file:} \par
The output file contains the same columns as the input file plus additional ones:
\begin{itemize}
        \item {\tt ValueString} --- String column containing the readback value as a ASCII
        string.  
        \item {\tt Lineage} --- String column containing the composite device
        name. In the case that a composite device was given as a control name in the input
        file, \verb+burtrb+ will expand the composite device in its atomic devices, and
        write one row per atomic device in the output file with the same {\tt Lineage}
        value for each.  If the control name in the input file was a process variable or
        an atomic device then the lineage is the character ``-''. (A quirk of program
        \verb+burtrb+ is the use of the character ``-'' for the equivalent of the empty
        string. A preferred way to specify a empty string would be simply ``'', but this
        hasn't been implemented in \verb+burtrb+.)  
        \item {\tt BackupMsg} --- String
        column. If not specified in the input file, the default device read message is
        written to this column.  
        \item {\tt RestoreMsg} --- String column. If not
        specified in the input file, the default device write message is written to this
        column.
\end{itemize}
Defined parameters in the output file are:
\begin{itemize}
        \item {\tt TimeStamp} --- String giving the date.
        \item {\tt BurtComment} --- String given at the comment command line option.
\end{itemize}
\end{itemize}

%
\item {\bf switches:}
%
% Describe the switches that are available
%
    \begin{itemize}
%
%   \item {\tt -pipe[=input][,output]} --- The standard SDDS Toolkit pipe option.
%
        \item {\tt -f file1 [file2] ...} --- Request files.
        \item {\tt -o outfile} --- Output file. If option is not present then stdout is the default.
        \item {\tt -c comment} --- A comment string to be added to the output file.
    \end{itemize}

\item {\bf see also:}
    \begin{itemize}
%
% Insert references to other programs by duplicating this line and 
% replacing <prog> with the program to be referenced:
%
    \item \progref{burtwb}
    \end{itemize}
%
% Insert your name and affiliation after the '}'
%
\item {\bf author: N. Karonis, ANL} 
\item {\bf manual page: L. Emery, ANL} 
\end{itemize}
