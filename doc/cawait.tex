% cawait.tex documentation for cawait utility
% template similar to others
\begin{sddsprog}{cawait}
\item \textbf{description:}
\verb+cawait+ monitors one or more EPICS process variables and waits until
user-specified conditions are met. Logical operations allow combining
conditions, and optional commands or subprocesses may be invoked when
events occur.
\item \textbf{examples:}
Wait for beam current to rise above 5~mA and echo a message:
\begin{flushleft}{\tt
cawait -waitFor=SR:current,above=5 -onEvent="echo beam reached 5"
}\end{flushleft}
Require two PVs to satisfy conditions simultaneously:
\begin{flushleft}{\tt
cawait -waitFor=SR:state,equalTo=1 -and -waitFor=SR:mode,equalTo=2
}\end{flushleft}
\item \textbf{synopsis:}
\begin{flushleft}{\tt
cawait [-interval=<seconds>] [-timeLimit=<seconds>] [-totalTimeLimit=<seconds>]\
 -waitFor=<PVname>[,lowerLimit=<value>,upperLimit=<value>][,equalTo=<value>][,sameAs=<string>][,above=<value>][,below=<value>][,changed][,monitor]\
 [{-and | -or | -not}]\
 [-repeat[=<number>]]\
 [-emit=event=<string>[,timeout=<string>][,end=<string>]]\
 [-preEvent=<command>]\
 [-onEvent=<command>]\
 [-postEvent=<command>]\
 [-onEnd=<command>]\
 [-subprocess=<command>[,event=<string>][,timeout=<string>][,end=<string>]]\
 [-pendIOTime=<seconds>]\
 [-noWarnings]\
 [-provider=\{ca|pva\}]
}\end{flushleft}
\item \textbf{switches:}
  \begin{itemize}
  \item {\tt -interval} --- seconds between checks (default 0.1~s).
  \item {\tt -timeLimit} --- maximum time to wait for the condition.
  \item {\tt -totalTimeLimit} --- maximum total runtime.
  \item {\tt -waitFor} --- PV name and criterion for ending wait.
  \item {\tt -and}, {\tt -or}, {\tt -not} --- logical operations on previous results.
  \item {\tt -repeat} --- number of events to wait for (default~1; infinite if omitted).
  \item {\tt -emit} --- emit strings for events, timeouts, or end-of-run.
  \item {\tt -preEvent}, {\tt -onEvent}, {\tt -postEvent}, {\tt -onEnd} --- commands to execute before, during, after events, or at end.
  \item {\tt -subprocess} --- run a subprocess and send event/timeout/end messages.
  \item {\tt -pendIOTime} --- time allowed for channel access operations (default 10~s).
  \item {\tt -noWarnings} --- suppress timeout warnings.
  \item {\tt -provider} --- choose channel access ({\tt ca}) or PVAccess ({\tt pva}) provider.
  \end{itemize}
\item \textbf{author:} M. Borland, R. Soliday, ANL/APS.
\end{sddsprog}

