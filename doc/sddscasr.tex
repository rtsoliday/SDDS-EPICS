%
% Template for making SDDS Toolkit manual entries.
%
\begin{latexonly}
\newpage
\end{latexonly}

%
% Substitute the program name for sddsmonitor
%
\subsection{sddscasr}
\label{sddscasr}

\begin{itemize}
\item {\bf description:}
%
% Insert text of description (typicall a paragraph) here.
%
\verb+sddscasr+ is an alternative version of casave and carestore. sddscasr is
more efficient and has more features than casave/carestore. It can also replace 
burtrb/burtwb for saving and restoring configurations.

\item {\bf example:} 
%
% Insert text of examples in this section.  Examples should be simple and
% should be proceeded by a brief description.  Wrap the commands for each
% example in the following construct:
% 
%
Save a snapshot of APS storage ring:
\begin{verbatim}
sddscasr SR.req SR.snapshot -save -pendIOTime=100
\end{verbatim}
Restore a snapshot:
\begin{verbatim}
sddscasr snapshot -restore -pendIOTime=100
\end{verbatim}
save snapshot with daemon mode, the output file in following command is out1-<current date and time>.
Whenever the value of casavePV (oag:casave) is changed to 1, a new saving starts and data is written
to a new output file with rootname of out1. (var1 is the input file contains PVs to be read)
\begin{verbatim}
sddscasr var1 out1 -runControlPV=string=oag:ControlLawRC 
-runControlDesc=string=test -daemon -daily -save -pidFile=pidFile
 -casavePV=oag:casave -logFile=logFile &
\end{verbatim}

\item {\bf synopsis:} 
%
% Insert usage message here:
%
\begin{verbatim}
usage: sddscasr <inputfile> <outputRoot> [-verbose]
    [-daemon] [-dailyFiles] [-semaphore=<filename>] [-save] [-restore] [-logFile=<filename>]
    [-runControlPV={string=<string>|parameter=<string>},pingTimeout=<value>,pingInterval=<value>]
    [-runControlDescription={string=<string>|parameter=<string>}] [-unique] [-outputFilePV=<pvname>]
    [-pidFile=<pidFile>] [-casavePV=<string>] [-interval=<seconds>] [-pipe=[input|output]]
    [-numerical] [-waveform=[rootname=<string>][,directory=<string>]]
    [-outputFilePV=<pvname>] [-casavePV=<string>]
\end{verbatim}
\item {\bf files:}
% Describe the files that are used and produced
\begin{itemize}
\item {\bf input file:} \par
The input file is an SDDS file with a string column. For saving snapshot, the input file
contains at least one string column - ControlName. For restoring snapshot, the input file
contains at least two strin columns - ControlName and ValueString, where ValueString is the
value of the PVs to be restored.
\item {\bf output file:} \par
The output file only exists for saving snapshot. The output file contains everything in the input
file, except that the ValueString column (if the input file has ValueString column) is update. And
three more columns are created if they do not exist in the input file and their values are updated.
\begin{itemize}
\item {ValueString}: the values of the PVs given in the input file.
\item {IndirectName}: its value  is - for scalar PV or PVname for waveform PV.
\item {CAError}: its value is y if error occurred during channel access for
corresponding PV or n if no error occurrs.
\end{itemize}
\end{itemize}

%
\item {\bf switches:}
%
% Describe the switches that are available
%
    \begin{itemize}
    \item {\tt inputFile} --- inputFile name (SDDS file).
    \item {\tt outputRoot} --- output file or root if -dailyFiles option is specified.
    \item {\tt pipe[=input][,output]} --- The standard SDDS Toolkit pipe option.
         -dailyFile option is ignored if -pipe=out or -pipe option is present.
    \item {\tt semaphore} --- specify the flag file for indicating CA connection completence.
               the current outputFile name is written to semaphore file.
    \item {\tt daemon} --- run in daemon mode, that is, stay in background and start 
          running whenever signal event arrived until terminating signal received.
    \item {\tt save} --- read the values of PVs given in the input file and write
               to the output file.
    \item {\tt restore} --- set the values of process variables given in the input file.
               -save and -restore can not be given in the same time.
    \item {\tt logFile} --- file for logging the printout information.
    \item {\tt pidFile} --- provide the file to write the PID into.
    \item {\tt interval} --- the interval (in seconds) of checking monitor pv 
               or the sleeping time if no signal handling arrived. 
    \item {\tt pendIOTime} --- specifies maximum time to wait for connections and
               return values. Default is 30.0s. It is important to give enough pendIOTime,
               otherwise, the CA connection could not set up and errors would occur. For SR
               request file, normally 100 seconds is approriate.
    \item {\tt casavePV} --- a monitor pv to turn on/off sddscasr. Whenever the casavePV is 
               set to 1, sddssave make a save or restore and change the pv back to 0.
    \item {\tt outputFilePV} --- a string pv to store the output file name (exclude the path).
    \item {\tt unique} --- remove all duplicates but the first occurrence of each PV from
               the input file to prevent channel access errors when setting pv values.
    \item {\tt description} --- specify SnapshotDescription parameter of output file.
    \item {\tt runControlPV=string=<string>|parameter=<string>[,pingTimeout=<value>,pingInterval=<value>]} --- specifies the runControl PV record.
    \item {\tt runControlDescription} --- specifies a string parameter whose value is
               a runControl PV description record.
    \item {\tt numerical} ---return a numerical string value rather than a string for enumerated types.
    \item {\tt waveform=rootname=<string>[,directory=<string>]} ---provide the waveform rootname and directory in restore option. By default, the directory is pwd, the rootname is the SnapshotFilename parameter in the input file.
    \item {\tt outputFilePV} --- a charachter-waveform pv to store the output file name with full path.
    \item {\tt casavePV} --- a monitor pv to turn on/off sddscasr. When the pv changed from 0 to 1, sddssave make a save and change the pv back to 0.
    
    \end{itemize}

\item {\bf see also:}
    \begin{itemize}
%
% Insert references to other programs by duplicating this line and 
% replacing <prog> with the program to be referenced:
%
    \item \progref{burtrb}
    \item \progref{burtwb}
    \end{itemize}
%
% Insert your name and affiliation after the '}'
%
\item {\bf author: H. Shang, ANL}
\end{itemize}
