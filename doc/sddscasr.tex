% sddscasr: save and restore PV configurations with enhanced features.
\begin{sddsprog}{sddscasr}
\item \textbf{description:}
\verb+sddscasr+ is an alternative version of casave and carestore. sddscasr is
more efficient and has more features than casave/carestore. It can save
and restore configurations in one program.

\item \textbf{examples:}
Save a snapshot of APS storage ring:
\begin{verbatim}
sddscasr SR.req SR.snapshot -save -pendIOTime=100
\end{verbatim}
Restore a snapshot:
\begin{verbatim}
sddscasr snapshot -restore -pendIOTime=100
\end{verbatim}
save snapshot with daemon mode, the output file in following command is out1-<current date and time>.
Whenever the value of casavePV (oag:casave) is changed to 1, a new saving starts and data is written
to a new output file with rootname of out1. (var1 is the input file contains PVs to be read)
\begin{verbatim}
sddscasr var1 out1 -runControlPV=string=oag:ControlLawRC
-runControlDescription=string=test -daemon -dailyFiles -save -pidFile=pidFile
-casavePV=oag:casave -logFile=logFile &
\end{verbatim}

\item \textbf{synopsis:}
\begin{verbatim}
usage: sddscasr <inputFile> <outputRoot> [-verbose]
    [-daemon] [-dailyFiles] [-semaphore=<filename>] [-save] [-restore[=verify]] [-logFile=<filename>]
    [-runControlPV={string=<string>|parameter=<string>},pingTimeout=<value>,pingInterval=<value>]
    [-runControlDescription={string=<string>|parameter=<string>}] [-unique]
    [-outputFilePV=<pvname>] [-inputFilePV=<pvname>] [-add] [-dryRun]
    [-pidFile=<pidFile>] [-casavePV=<string>] [-interval=<seconds>] [-pendIOTime=<seconds>]
    [-pipe=[input|output]] [-waveform=[rootname=<string>][,extension=<string>][,directory=<string>][,onefile][,saveWaveformFile[,fullname]]]
    [-numerical] [-description=<string>|pv=<pvname>]
\end{verbatim}
\item \textbf{files:}
\begin{itemize}
  \item \textbf{input file:} \par
The input file is an SDDS file with a string column. For saving snapshot, the input file
contains at least one string column - ControlName. For restoring snapshot, the input file
contains at least two string columns - ControlName and ValueString, where ValueString is the
value of the PVs to be restored.
  \item \textbf{output file:} \par
The output file only exists for saving snapshot. The output file contains everything in the input
file, except that the ValueString column (if the input file has ValueString column) is update. And
three more columns are created if they do not exist in the input file and their values are updated.
  \begin{itemize}
    \item {ValueString}: the values of the PVs given in the input file.
    \item {IndirectName}: its value  is - for scalar PV or PVname for waveform PV.
    \item {CAError}: its value is y if error occurred during channel access for
corresponding PV or n if no error occurs.
  \end{itemize}
\end{itemize}

\item \textbf{switches:}
\begin{itemize}
  \item {\tt inputFile} --- input file name (SDDS file).
  \item {\tt outputRoot} --- output file or root if -dailyFiles option is specified.
  \item {\tt pipe[=input][,output]} --- The standard SDDS Toolkit pipe option.
        -dailyFiles option is ignored if -pipe=out or -pipe option is present.
  \item {\tt semaphore} --- specify the flag file for indicating CA connection completion;
        the current output file name is written to the semaphore file.
  \item {\tt daemon} --- run in daemon mode, staying in the background until a terminating signal is received.
  \item {\tt dailyFiles} --- append the current date to the output file name.
  \item {\tt save} --- read the values of PVs given in the input file and write to the output file.
  \item {\tt restore[=verify]} --- set the values of process variables given in the input file.
        -save and -restore cannot be given at the same time.
  \item {\tt add} --- with -restore, set PVs to current value plus provided value.
  \item {\tt dryRun} --- with -restore, show PV values that would be set without changing them.
  \item {\tt logFile} --- file for logging the printout information.
  \item {\tt pidFile} --- file to which the PID is written.
  \item {\tt interval} --- interval (in seconds) for checking the monitor PV or sleeping if no signal arrives.
  \item {\tt pendIOTime} --- maximum time to wait for connections and return values.
  \item {\tt casavePV} --- a monitor PV to turn on/off sddscasr. When set to 1, a save or restore is made and the PV is reset to 0.
  \item {\tt outputFilePV} --- a character-waveform PV to store the output file name with full path.
  \item {\tt inputFilePV} --- a character-waveform PV containing the input file name with full path.
  \item {\tt unique} --- remove all duplicates but the first occurrence of each PV from the input file to prevent channel access errors when setting PV values.
  \item {\tt description} --- specify SnapshotDescription parameter of the output file or provide a PV using {\tt pv=<pvname>}.
  \item {\tt runControlPV=string=<string>|parameter=<string>[,pingTimeout=<value>,pingInterval=<value>]} --- specifies the run control PV record.
  \item {\tt runControlDescription} --- specifies a string parameter whose value is a run control PV description record.
  \item {\tt numerical} --- return a numerical string value rather than a string for enumerated types.
  \item {\tt waveform=rootname=<string>[,extension=<string>][,directory=<string>][,onefile][,saveWaveformFile][,fullname]} --- provide the waveform rootname and directory for the restore option. By default, the directory is pwd and the rootname is the SnapshotFilename parameter in the input file.
  \item {\tt verbose} --- print out messages.
\end{itemize}

\item \textbf{see also:}
\begin{itemize}
  \item \progref{sddssnapshot}
  \item \progref{sddscaramp}
\end{itemize}
\item \textbf{author:} H. Shang, ANL
\end{sddsprog}
