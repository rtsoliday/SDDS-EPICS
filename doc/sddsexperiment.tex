%
% Template for making SDDS Toolkit manual entries.
%
\begin{sddsprog}{sddsexperiment}
\item {\bf description:}
%
% Insert text of description (typically a paragraph) here.
%
\verb+sddsexperiment+ varies process variables and measures process variables, with optional statistical analysis.
An input file of namelist commands gives the specific instructions. 
The results are recorded in one or more SDDS files.
\item {\bf example:} 
%
% Insert text of examples in this section.  Examples should be simple and
% should be proceeded by a brief description.  Wrap the commands for each
% example in the following construct:
% 
%
The strength of a beamline horizontal corrector (LTP:H1) is varied while the readbacks at a horizontal beam position
monitor (LTP:PH1) are recorded. The output file is LTP:H1.sdds.
\begin{verbatim}
sddsexperiment LTP:H1.exp LTP:H1.sdds
\end{verbatim}
where the contents of the file \verb+LTP:H1.exp+ are
\begin{verbatim}
&measurement    control_name = "LTP:PH1:x",
         column_name="LTP:PH1:x", units=mm,
         number_to_average = 10,
&end

&variable control_name = "LTP:H1:CurrentAO",
        column_name="LTP:H1:CurrentAO"
! the corrector is varied in 5 steps from -1.0 to 1.0 amps.
        index_number = 0, index_limit = 5,
        initial_value = -1.0, final_value = 1.0,
&end
        
&execute
        post_change_pause=4,
        intermeasurement_pause=1
&end
\end{verbatim}

where the line starting with a ``!'' is a comment.
\item {\bf synopsis:} 
%
% Insert usage message here:
%
\begin{verbatim}
usage: sddsexperiment <inputfile> [<outputfile-rootname|outputfilename>]
[-echoinput] [-dryrun] [-summarize] [-verbose] [-pendIOtime=<seconds>]
[-describeInput] [-macro=<tag>=<value>,[...]] [-comment=<string value>] [-scalars=<filename>]
\end{verbatim}

\item {\bf files:}
\begin{itemize}
\item {\bf input file:}\par
The input file consists of namelist commands that set up and execute the experiment. The functions of the commands
are described below.
\begin{itemize}
        \item {\verb+variable+} --- specifies a process variable to vary, and the range and steps of the variation. More than one
                variable command may be defined, so that many process variables may vary at a time.
                When an arbitrary sequence of setpoint values is required, the setpoints can be read in 
                from a file.
        \item {\verb+measurement+} --- specifies a process variable to measure at each step during the experiment.
                Instead of using many {\verb+measurement+} namelist, 
                one can optionally use a sddsmonitor-compatible file.
        \item {\verb+execute+} --- start executing the experiment. One group of variable, measurement and execute commands
                may follow another in the same file for multiple experiments.
        \item {\verb+erase+} --- deletes previous variable or measurement setups.
        \item {\verb+list_control_quantities+} --- makes a cross-reference file for process variable names and column names of the data file.
        \item {\verb+system_call+} --- specifies a system call (usually a script) to be executed either before a measurement or before
                setting a process variable.
\end{itemize}

The following text describes all the namelist commands and their respective fields in more detail. 
The command definition listing is of the form
\begin{verbatim}
&<command-name>
    <variable-type> <variable-name> = <default-value>
    .
    .
    .
&end
\end{verbatim}
where the part \verb+<variable-type>+, which doesn't appear in an actual command, is used to illustrate
the valid type of the value. The three valid types are:
\begin{itemize}
\item \verb+double+ ---  for a double-precision type, used for most physical quantity variables,
\item \verb+long+ ---   for an integer type, used for indexes and flags.
\item \verb+STRING+ --- for a character string enclosed in double quotes.
\end{itemize}
An actual namelist in an input file should look like this:
\begin{verbatim}
&<command-name>
    [<variable-name> = <value>,]
    ...
&end
\end{verbatim}

In the namelist definition listings the square brackets denotes an optional component.
Not all variables need to be defined -- the defaults may be sufficient.
Those that do need to be defined are noted in the detailed explanations.
The only variables that don't have default values in general are string variables.

%%%%%%%%%%%%%%%%%%%%%%%%%%%%%%%%%%%%%%%%%%%%%%%%%%%%%%%%%%%%%%%%%%%%%%%%%%%%%%%%
% this extra level of itemization is required to make the html output look
% good.
%%%%%%%%%%%%%%%%%%%%%%%%%%%%%%%%%%%%%%%%%%%%%%%%%%%%%%%%%%%%%%%%%%%%%%%%%%%%%%%%
%\begin{htmlonly}
%\begin{itemize}
%\end{htmlonly}

\begin{latexonly}
\newpage\begin{center}{\Large \verb+variable+}\end{center}
\end{latexonly}
\begin{htmlonly}
\item {\Large \verb+variable+}
\end{htmlonly}
\begin{itemize}
        \item function: Specifies a process variable to vary, and the range and steps of the variation.
        Values of variables at each measurement step are written to an SDDS output file.
        The readback-related fields are used to confirm that the physical device has responded 
        to a setpoint command at every step (and substep) within some tolerance. Readback is enabled
        when {\verb+readback_attempts+} and {\verb+readback_tolerance+} are defined with non-zero positive values.

        When an arbitrary sequence of setpoint values is required (say a binary sequence), 
        the values can be read in from an SDDS file specified by the {\verb+values_file+} field. 
        The fields of \verb+&variable+ associated for the range and steps are ignored in this case.

         With multiple \verb+variable+ commands,
         variables may be varied in a multi-dimensional grid. For example,
        variables may be varied independently of each other, or some groups of
        variables may vary together forming one axis of a multi-dimensional grid
         (see item {\verb+index_number+}). 

\begin{verbatim}
&variable
        STRING control_name = NULL
        STRING column_name = NULL
        STRING symbol = NULL
        STRING units = "unknown"
        double initial_value = 0
        double final_value = 0
        long relative_to_original = 0
        long index_number = 0
        long index_limit = 0
        STRING function = NULL
        STRING values_file = NULL
        STRING values_file_column = NULL
        long substeps = 1
        double substep_pause = 0
        double range_multiplier = 1
        STRING readback_name = NULL
        double readback_pause = 0.1
        double readback_tolerance = 0
        long readback_attempts = 10
        long reset_to_original = 1
&end        
\end{verbatim}
        \item {\verb+control_name+} --- Required. Process variable name to vary. 
        \item {\verb+column_name+} --- Required. Column name for the variable data recorded in the output file.
        \item {\verb+symbol+} --- Optional. Symbol field for the above column definition of the variable data.
        \item {\verb+units+} --- Optional. Units field for the above column definition of the variable data.
        \item {\verb+initial_value+} --- Required. The initial value of the process variable in the variation.
        \item {\verb+final_value+} --- Required. The final value of the process variable in the variation.
        \item {\verb+index_number+} --- Required. The counter (or index) number with which
                the defined variation is associated. In a \verb+sddsexperiment+ run, counters must
                be defined in an increasing order with no gaps starting from counter 0. That is,
                the first {\verb+variable+} command of the 
                file must have {\verb+index_number+} = 0. The second {\verb+variable+} command
                must have {\verb+index_number+} = 0 or 1. In the former case, the two variables
                will move together with the same number of steps according their respective
                {\verb+initial_value+} and {\verb+final_value+}. In the latter
                case, the two variables will vary independently of each other with possibly different
                number of steps in a 2-dimensional grid.

                Counter number $n$ is nested within counter $n+1$. Therefore it might be efficient
                to assign devices with slower response times to
                higher \verb+index_number+ counter.
        \item {\verb+index_limit+} --- Normally required. 
                Number of steps in the variation. Measurements are taken at each step. 
                When more than one variable is associated with the same counter, only the {\verb+index_limit+}
                of the first variable definition for that counter need to be defined.
                If {\verb+index_limit+} is defined in {\verb+variable+} commands
                of the same  {\verb+index_number+} value, then the first {\verb+index_limit+}
                remain in force.
        \item {\verb+relative_to_original+} --- Optional. If non-zero, then the variation range is defined 
                relative to the original process variable value (i.e. the value prior to running the program).
        \item {\verb+range_multiplier+} --- Optional. Factor by which the range, {\verb+final_value+} - {\verb+initial_value+}, is multiplied.
                New values of {\verb+initial_value+} and {\verb+final_value+} are calculated while keeping the midpoint of
                the range the same.
        \item {\verb+function+} --- Optional. A string of rpn operations used to transform the range specified 
                by {\verb+initial_value+}, {\verb+final_value+}, and {\verb+index_limit+}. 
                For convenience, a few values are pushed onto the stack and are available
                to the user: the original value of the process variable, and the 
                calculated grid value for the process variable on the current step or substep. 
                The calculated values are
                recorded in the output file. The environment variable \verb+RPN_DEFNS+
                is used to read a rpn definition file at the start of the execution of \verb+sddsexperiment+.
        \item {\verb+values_file+} --- Optional. An SDDS data file containing setpoints for the variable.
                This is useful is one has arbitrary setpoints values to apply.
                The values of the fields {\verb+initial_value+}, {\verb+final_value+}, {\verb+_substeps+}, 
                {\verb+range_multiplier+} and {\verb+index_limit+} are ignored. 

                One can have other \verb+variable+ namelists with the same \verb+index_number+
                that don't use a file for the values.
                The default {\verb+index_limit+} of the other variable will be
                set to the number of setpoint in the values file.
                Thus the values in the file and the values calculated for the other variable 
                will vary together with the same number of steps.
        \item {\verb+values_file_column+} --- Required when {\verb+values_file+} is specified.
                {\verb+values_file_column+} gives the column name of the setpoints data 
                in file {\verb+values_file+}. 
        \item {\verb+substeps+} --- Optional. If greater than one, the steps are subdivided into this number.
                Measurements are not made at substeps. Substeps are useful
                when the physical device associated with the process variable cannot reliably make 
                steps as large as those that might be defined with {\verb+initial_value+},
                {\verb+final_value+}, and {\verb+index_limit+}.
        \item {\verb+substep_pause+} --- Optional. Number of seconds to pause after the variable change of each substeps.
        \item {\verb+readback_name+} --- Optional. Readback process variable name 
                associated with {\verb+control_name+}. 
                The default value for {\verb+readback_name+} is {\verb+control_name+}.
        \item {\verb+readback_tolerance+} --- Optional. Maximum acceptable
                absolute value of the difference between the process variable
                setpoint and its readback. A positive value is required in order to enable readbacks.
        \item {\verb+readback_pause+} --- Optional. Number of seconds to pause after 
                each reading of the {\verb+readback_name+} process variable. 
                This pause time is in addition to other pauses defined.
        \item {\verb+readback_attempts+} --- Optional. Number of allowed readings 
                of the {\verb+readback_name+} process variable
                and readback pauses after a variable change has occurred. 
                After this number of readings, the program exits.
                The first readback is attempted immediately (i.e. no pause) after 
                sending a setpoint command to the {\verb+control_name+}.
                A positive value is required in order to enable readbacks.
        \item {\verb+reset_to_original+} --- Optional. A value of 1 means 
                that the variable is reset to its original value when the 
                experiment terminates normally or abnormally. 
\end{itemize}

\begin{latexonly}
\newpage\begin{center}{\Large \verb+measurement_file+}\end{center}
\end{latexonly}
\begin{htmlonly}
\item {\Large \verb+measurement_file+}
\end{htmlonly}
\begin{itemize}
        \item function: specifies a SDDS data file containing the names of
                 the process variables to measure at each step during the experiment.
\begin{verbatim}
&measurement_file
        STRING filename = NULL
        long number_to_average = 1
        long include_standard_deviation = 0
        long include_sigma = 0
        double lower_limit = 0
        double upper_limit = 0
        long limits_all = 0
&end
\end{verbatim}
        \item {\verb+filename+} --- Required. SDDS file containing measurement process variables.
                The required and optional column definitions closely resembles those used in {\verb+sddsmonitor+}.
                The columns are:
        \begin{itemize}
                \item {\verb+ControlName+} --- Required string data for the process variable names.
                \item {\verb+ReadbackName+} --- Optional string data for the measurement data
                columns names of the experiment output file. If the column is not present, then the
                experiment output file's columns names uses the measurement process variable names themselves.
                \item {\verb+NumberToAverage+} --- Optional long data giving the number of measurements
                to average at each step of the experiment. The average value of the measurements for each
                process variable is written to the output file. Therefore individual readings are not recorded.
                \item {\verb+IncludeStDev+} --- Optional long data. If non-zero, then the standard
                deviations of measurements are calculated and a column in the output file for this quantity
                is generated.
                \item {\verb+IncludeSigma+} --- Optional long data.  If non-zero, then the sigma
                of measurements are calculated and a column in the output file for this quantity
                is generated.
                \item {\verb+LowerLimit+}, {\verb+UpperLimit+} --- Optional double data. Must have
                both or neither. Specifies a range outside of which the measurement
                data is to be rejected, and the measurement be retaken.
                If the number of invalid measurements (reset to 0 at each
                measurement step) equals or exceeds the value
                of {\verb+allowed_limit_errors+} (default of 1) in command {\verb+execute+}, 
                then the program aborts.
                The average values written to the output file excludes measurements outside this range.
                \item {\verb+LimitsAll+} --- Optional long data. If non-zero for a particular PV, then 
                out-of-range data for this PV causes all other PVs to be remeasured. 
                By default, only the out-of-range PV is remeasured.
        \end{itemize}
                Only the first data page of file {\verb+filename+} is read in. For those
                optional columns above that are not defined, then the following {\verb+measurement_file+} fields
                will act as defaults (note the different spellings):{\verb+number_to_average+}, 
                {\verb+include_standard_deviation+}, {\verb+include_sigma+}, 
                {\verb+lower_limit+}, {\verb+upper_limit+}, {\verb+limits_all+}.
\end{itemize}


\begin{latexonly}
\newpage\begin{center}{\Large \verb+measurement+}\end{center}
\end{latexonly}
\begin{htmlonly}
\item {\Large \verb+measurement+}
\end{htmlonly}
\begin{itemize}
        \item function: specifies a process variable to measure at each step during the experiment.
                This command is compatible with {\verb+measurement_file+}, as both commands
                merely adds to an internal list of measurement PV.
\begin{verbatim}
&measurement
        STRING control_name = NULL
        STRING column_name = NULL
        STRING symbol = NULL
        STRING units = "unknown"
        long number_to_average = 1
        long include_standard_deviation = 0
        long include_sigma = 0
        double lower_limit = 0
        double upper_limit = 0
        long limits_all = 0
&end
\end{verbatim}
        \item {\verb+control_name+} --- Required. Process variable name to measure.
        \item {\verb+column_name+} --- Required. Column name for the measurement data recorded in the output file.
        \item {\verb+symbol+} --- Optional. Symbol field for the above column definition of the measurement data.
        \item {\verb+units+} --- Optional. Units field for the above column definition of the measurement data.
        \item {\verb+number_to_average+} --- Optional. Number of readings to average. The output data
                is the average value. The individual readings are not recorded.
        \item {\verb+include_standard_deviation+} ---  Optional. Calculate the standard deviation 
                (a measure of the distribution) of the
                measurement, and generate a column in the output file for this quantity.
        \item {\verb+include_sigma+} ---  Optional. Calculate the sigma (error of the mean) of the
                measurement, and generate a column in the output file for this quantity.
        \item {\verb+lower_limit+}, {\verb+upper_limit+} --- Optional. Defines a range of validity for the
                individual measurements. If the number of invalid measurements (reset to 0 at each
                measurement step) equals or exceeds the value
                of {\verb+allowed_limit_errors+} (default of 1) in command {\verb+execute+}, then the program aborts.
                The average values written to the output file excludes measurements outside this range.
        \item {\verb+limits_all+} --- Optional. If non-zero, then out-of-limits data for this PV causes 
                all PVS to be remeasured. By default, only the out-of-limit PV is remeasured.
\end{itemize}

\begin{latexonly}
\newpage\begin{center}{\Large \verb+execute+}\end{center}
\end{latexonly}
\begin{htmlonly}
\item {\Large \verb+execute+}
\end{htmlonly}
\begin{itemize}
        \item function: start executing the experiment. Some global parameters are defined here.
\begin{verbatim}
&execute
        STRING outputfile = NULL
        double post_change_pause = 0
        double intermeasurement_pause = 0
        double rollover_pause = 0
        long post_change_key_wait = 0
        long allowed_timeout_errors = 1
        long allowed_limit_errors = 1
        double outlimit_pause = 0.1
        long repeat_reading = 1
        double post_reading_pause = 0.1
        double ramp_pause = 0.25
        long ramp_steps = 10
&end
\end{verbatim}
        \item {\verb+outputfile+} --- Optional. SDDS output file of variable and measurement data. The string
                may contain the string ``\%s'' which is substituted with the rootname value
                of the command line. If rootname is not given, then the root of the input file
                is substituted . If {\verb+outputfile+} is null, then the root of the input file
                is used for the output file. See description of output file below.
        \item {\verb+post_change_pause+} ---  Optional. Number of seconds to pause after each change before
                making measurements.
        \item {\verb+intermeasurement_pause+} ---  Optional. Number of seconds to pause between each measurement.
                Individual measurements for averaging are taken at this interval.
        \item {\verb+rollover_pause+} ---  Optional. Number of seconds to pause after a counter has reached
                its upper limit, and must rollover to zero. This allows any physical devices
                associated with the counter to settle after a change equal to the total range
                of the variation.
        \item {\verb+post_change_key_wait+} ---  Optional. If non-zero, then wait for a key press after
                making variable changes but before taking measurements. A prompt is given.
        \item {\verb+allowed_timeout_errors+} ---  Optional. Number of timeout ezca errors allowed before aborting the
                program. 
        \item {\verb+allowed_limit_errors+} ---  Optional. Number of invalid range measurement errors 
                allowed before aborting the
                program. The valid range of a measurement is specified in the {\verb+measurement+} command.
        \item {\verb+outlimit_pause+} ---  Optional. Number of seconds to pause after an invalid range measurement error
                occurred. This is to permit equipment time to recover from whatever glitch caused the out-of-limit
                reading.
        \item {\verb+repeat_reading+} --- Optional. The measurements and statistical analyses are repeated this number of 
                times for each variable settings. A row of data is written to the output file
                for each repetition.
        \item {\verb+post_reading_pause+} --- Optional. Number of seconds to pause after taking a 
                set of measurements and making a statistical analysis.
                If measurements are repeated then the pause is repeated after each set of measurements.
        \item {\verb+ramp_steps+} --- Optional. Number of steps in the variables PV ramp 
                which occurs at the start and the end of the experiment.

                Ramping is necessary for some devices that do not respond well
                to large changes to their setpoints. Ramping is done
                at the start of the experiments to slowly change the variable PVs from their
                current values to their initial values. Another ramp is
                done at the end to slowly bring the variable PVs from their final values
                back the original values. Ramping back to original values
                is also done when the experiment aborts for some reason.
        \item {\verb+ramp_pause+} --- Optional. Time interval at each step of the variables PV ramp 
                which occurs at the start and the end of the experiment. This is not the same variable
                as the pause between variable changes during the experiment.
\end{itemize}

\begin{latexonly}
\newpage\begin{center}{\Large \verb+erase+}\end{center}
\end{latexonly}
\begin{htmlonly}
\item {\Large \verb+erase+}
\end{htmlonly}
\begin{itemize}
        \item function: deletes previous variable or measurement setups.
\begin{verbatim}
&erase
        long variable_definitions = 1
        long measurement_definitions = 1
&end
\end{verbatim}
        \item {\verb+variable_definitions+} --- Optional. If non-null, then all the variable definitions are erased.
        \item {\verb+measurement_definitions+} --- Optional. If non-null, then all the measurement definitions are erased.
\end{itemize}

\begin{latexonly}
%\newpage
\begin{center}{\Large \verb+list_control_quantities+}\end{center}
\end{latexonly}
\begin{htmlonly}
\item {\Large \verb+list_control_quantities+}
\end{htmlonly}
\begin{itemize}
        \item function: makes a cross-reference file for process variable names and column names of the data file.
\begin{verbatim}
&list_control_quantities
         STRING filename = NULL
&end
\end{verbatim}
        \item {\verb+filename+} --- Required. Name of file. Columns defined are \verb+ControlName+,
               \verb+SymbolicName+, and \verb+ControlUnits+.
\end{itemize}

\begin{latexonly}
\newpage\begin{center}{\Large \verb+system_call+}\end{center}
\end{latexonly}
\begin{htmlonly}
\item {\Large \verb+system_call+}
\end{htmlonly}
\begin{itemize}
        \item function: specifies a system call (usually a script) to be executed repeatedly during the experiment.
\begin{verbatim}
&system_call
        STRING command = NULL
        long index_number = 0
        long index_limit = 0
        double post_command_pause = 0
        double pre_command_pause = 0
        long append_counter = 0
        STRING counter_format = "%ld"
        long call_before_setting = 0
        long call_before_measuring = 1
        STRING counter_column_name = NULL 
&end       
\end{verbatim}
        \item {\verb+command+} --- Required. Name of shell command or script to execute.
        \item {\verb+index_number+} --- Required. Counter number with which the command will be associated. The command is executed
                when this counter is advanced or rolled over.
        \item {\verb+index_number+} --- Optional. Number of times the command is executed for 
                the associated counter. This field is used only when the value of {\verb+index_number+} above defines a new counter.
        \item {\verb+post_command_pause+} --- Optional. Number of seconds to pause after the completion of the command.
        \item {\verb+pre_command_pause+} --- Optional. Number of seconds to pause before executing the command.
        \item {\verb+append_counter+} --- Optional. If non-zero, the counter value is appended to the command when the
                system call is made.
        \item {\verb+counter_format+} --- Optional. Format for the counter if the counter value is appended to the command.
        \item {\verb+call_before_setting+}, {\verb+call_before_measuring+} --- Optional. 
                At a counter advance or rollover the command can be executed in one of three ways:
                \begin{itemize}
                        \item  before both variable changes and measurements: \\
                        {\verb+call_before_setting+}=1, {\verb+call_before_measuring+}=1
                        \item  after variable changes and before measurements:\\
                        {\verb+call_before_setting+}=0, {\verb+call_before_measuring+}=1
                        \item  after both variable changes and measurements:\\
                        {\verb+call_before_setting+}=0, {\verb+call_before_measuring+}=0
                \end{itemize}
                If multiple measurements are required for averaging, the command is not executed between these measurements.
        \item {\verb+counter_column_name+} --- Optional. If non-null, a column in the output file with this name is defined.
                The values written to this column are the number of times the command had been called minus one. This
                value doesn't rollover with its associated counter.
\end{itemize}

%\begin{htmlonly}
%\end{itemize}
%\end{htmlonly}

\begin{latexonly}
\newpage\begin{center}{\Large \verb+exec_command+}\end{center}
\end{latexonly}
\begin{htmlonly}
\item {\Large \verb+exec_command+}
\end{htmlonly}
\begin{itemize}
        \item function: executes one command one time. This is useful to the initial set-up of an experiment.
\begin{verbatim}
&exec_command
        STRING command = NULL
&end
\end{verbatim}
        \item {\verb+command+} --- Command or script to run. The command is run immediately when the namelist command is processed
\end{itemize}

\newpage
\item {\bf output file:}\par
The output file contains one data column for each measurement and control process variable.
The names of these data columns are taken from the \verb+column_name+ field of the
respective \verb+measurement+ and \verb+variable+ commands. The data are written
as double precision floating point numbers. 
In addition, some time columns and parameters are defined:
\begin{itemize}
        \item {\tt Time} --- Double column of time since start of epoch. This time data can be used by
        the plotting program {\verb+sddsplot+} to make the best choice of time unit conversions
        for time axis labeling.
        \item {\tt ElapsedTime} --- Double column of elapsed time of readback since the start of the experiment.
        \item {\tt TimeOfDay} --- Double column of system time in units of hours. 
                The time does not wrap around at 24 hours.
        \item {\tt TimeStamp} --- String parameter of time stamp for file.
\end{itemize}

\item {\bf control quantities list file:}\par
        This file contains process variable names, readback column names, and units as string data. 
        This data can be used for cross-referencing. The columns defined in this file are:
\begin{itemize}
        \item {\tt ControlName} --- String column. Value of \verb+control_name+ field given in the 
                \verb+measurement+ and \verb+variable+ commands.
        \item {\tt SymbolicName} --- String column. Value of \verb+column_name+ field given in the 
                \verb+measurement+ and \verb+variable+ commands.
        \item {\tt ControlUnits} --- String column.  Value of \verb+units+ field given in the 
                \verb+measurement+ and \verb+variable+ commands.
\end{itemize}
\end{itemize}


\item {\bf switches:}
%
% Describe the switches that are available
%
    \begin{itemize}
    \item {\verb+-echoinput+} --- echos input file to stdout.
    \item {\verb+-dryrun+} --- the ``variable'' process variables are left untouched during the execution. The ``measurement''
                process variables are still read. The pauses are still in effect. System calls are not made.
    \item {\verb+-summarize+} --- gives a summary of the experiment before executing it.
    \item {\verb+-verbose[=very]+} --- prints out information during the execution such as notification of 
                setting and reading process variables. The option \verb+very+ prints out the average measurement values.
    \item {\verb+-pendIOTime=<timeout>+} --- sets CA timeout in seconds.
    \item {\verb+-describeinput+} --- Printouts the list of namelist commands and fields of the input file.
    \item {\verb+-macro=<tag>=<value>[,...]+} --- replaces the macros (<tag>) in the input file by actual values (<value>).
    \item {\verb+-comment=<string value>+} --- provides the comment, which will be written to the output file as parameter.
    \item {\verb+-scalars=<filename>+} --- reads additional scalar process variables from the specified file.
    \end{itemize}

\item {\bf see also:}
    \begin{itemize}
%
% Insert references to other programs by duplicating this line and 
% replacing <prog> with the program to be referenced:
%
    \item \progref{sddsvexperiment}
    \end{itemize}
%
% Insert your name and affiliation after the '}'
%
\item {\bf author: M. Borland, L. Emery, H. Shang and R. Soliday ANL} 
\end{sddsprog}
