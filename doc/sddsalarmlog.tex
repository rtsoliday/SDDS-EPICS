%
% Template for making SDDS Toolkit manual entries.
%
\begin{latexonly}
\newpage
\end{latexonly}

%
% Substitute the program name for sddsalarmlog
%
\subsection{sddsalarmlog}
\label{sddsalarmlog}

\begin{itemize}
\item {\bf description:}
%
% Insert text of description (typically a paragraph) here.
%
\verb+sddsalarmlog+ logs changes of alarm status for process variables named in an input file.
\item {\bf example:} 
%
% Insert text of examples in this section.  Examples should be simple and
% should be proceeded by a brief description.  Wrap the commands for each
% example in the following construct:
% 
%
The status of the positron accumulator ring process variables are logged
for a one day duration.
\begin{verbatim}
sddsalarmlog PAR.along  PAR.along.sdds -timeDuration=1,day 
\end{verbatim}
where the file \verb+PAR.along+ is the input file containing the names of the
process variables, and the file \verb+PAR.along.out+ is the output file.

\item {\bf synopsis:} 
%
% Insert usage message here:
%
\begin{verbatim}
usage: sddsalarmlog <input> <output> 
-timeDuration=<realValue>[,<time-units>]
[{-append[=recover] | -eraseFile | -generations[=digits=<integer>][,delimiter=<string>} |
 -dailyFiles[=verbose]]
[-pendEventTime=<seconds>] [-durations] [-explicit[=only]] [-verbose]
[-comment=<parameterName>,<text>] 
\end{verbatim}

\item {\bf files:}
% Describe the files that are used and produced
\begin{itemize}
\item {\bf input file:}\par
The input file is an SDDS file with the required string column {\tt ControlName}
which contains the names of the process variables whose status is to be monitored.

\item {\bf output file:}\par
The output file contains information on the process variable alarm status
change, such as the time of occurrence, the alarm designation, the severity of the alarm, and the
row location of the previous alarm status change.

In order to save disk space {\tt sddsalarmlog} logs integer codes
(indexes) instead of the actual string values for the control name,
the alarm status and the alarm severity. The integer codes are indices
into one of three string arrays written to the output file. The logging information
can be recovered using SDDS tool sddsderef. 

As an option, and for a direct interpretation of the output file, the control
name, alarm status and alarm severity can be written explicitly as
string columns with or without the integer codes. However this
option uses up a lot more disk space.

In either case the following columns are defined:
\begin{itemize}
        \item {\tt PreviousRow} --- Long column of row numbers. For each 
        process variables alarm status change, the row location of the 
        previous alarm status change is written.                
        \item {\tt TimeOfDay} --- Float column of system time in units of hours. 
        The time does not wrap around at 24 hours.
\end{itemize}

By default these columns are defined (except when the option {\tt -explicit=only} is specified):
\begin{itemize}
        \item {\tt ControlNameIndex} --- Long column indicating the process variable whose
        alarm status changed. The value of this data is the index into the string array 
        {\tt ControlNameString}, which is the list of all process variables monitored.
        \item {\tt AlarmStatusIndex} --- Short column indicating the alarm status.
         The value of this data is the index into the string array 
        {\tt AlarmStatusString}, which is the list of all possible alarm status values.
        \item {\tt AlarmSeverityIndex} --- Short column indicating the alarm status.
        The value of this data is the index into the string array 
        {\tt AlarmSeverityString}, which is the list of all possible alarm severity values.
\end{itemize}

These columns are created by the option {\tt -explicit}:
\begin{itemize}
        \item {\tt ControlName} --- String column for the process variable whose alarm
        status just changed.
        \item {\tt AlarmStatus} --- String column for the alarm status. 
        \item {\tt AlarmSeverity} --- String column for the alarm severity. 
\end{itemize}

This column is created with option {\tt -durations}:
\begin{itemize}
        \item {\tt Duration} --- String column for the duration of the previous alarm state.
\end{itemize}

These arrays are created by default except when the option {\tt -explicit=only} is specified:
\begin{itemize}
        \item {\tt ControlNameString} --- String colunm of all process variables to be monitored.
        \item {\tt AlarmStatusString} --- String colunm of all possible alarm status values.
        \item {\tt AlarmSeverityString} --- String colunm of all possible alarm severity values.
\end{itemize}

These parameters are defined:
\begin{itemize}
        \item {\tt InputFile} --- String parameter for the name of the input file.
        \item {\tt TimeStamp} --- String parameter for time stamp for file.
        \item {\tt PageTimeStamp} --- String parameter for time stamp for each page. When data
                is appended to an existing file, the new data is written to a new
                page. The {\tt PageTimeStamp} value for the new page is the creation
                date of the new page. The {\tt TimeStamp} value for the new page is the creation 
                date of the very first page.
        \item {\tt StartTime} --- Double parameter for start time from the {\tt C} time call cast to type double.
        \item {\tt YearStartTime} --- Double parameter for start time of present year from the {\tt C} time call cast to type double.
        \item {\verb+StartYear+} --- Short parameter for the year when the file was started.
        \item {\verb+StartJulianDay+} --- Short parameter for the day when the file was started.
        \item {\verb+StartMonth+} --- Short parameter for the month when the file was started.
        \item {\verb+StartDayOfMonth+} --- Short parameter for the day of month when the file was started.
        \item {\verb+StartHour+} --- Short parameter for the hour when the file was started.
\end{itemize}
\end{itemize} % end of itemized files


%
\item {\bf switches:}
%
% Describe the switches that are available
%
    \begin{itemize}
%
%   \item {\tt -pipe[=input][,output]} --- The standard SDDS Toolkit pipe option.
%
        \item {\tt -timeDuration=<realValue>[,<time-units>]} ---  Specifies time duration for logging.
                The default time units are seconds; one may also specify days, hours, or minutes.
        \item {\tt -append[=recover]} ---Specifies appending to the file <output> if it exists already.
                If the recover qualifier is given, recovery of a corrupted file
                is attempted using sddsconvert, at the risk of loss of some of the
                data already in the file.
        \item {\tt -eraseFile} --- If the output file already exists, then it will be overwritten
                by \verb+sddsalarmlog+.
        \item {\verb+-generations[=digits=<integer>][,delimiter=<string>]+} ---
                The output is sent to the file \verb+<SDDSoutputfile>-<N>+, where \verb+<N>+ is
                   the smallest positive integer such that the file does not already
                   exist.   By default, four digits are used for formatting \verb+<N>+, so that
                   the first generation number is 0001.
        \item {\tt -pendEventTime=<seconds>} --- Specifies the CA pend event time, in seconds.
                The default is 10 .
        \item {\tt -durations} --- Specifies including state duration in output.
        \item {\tt -connectTimeout} --- Specifies maximum time in seconds to wait for a connection before
                issuing an error message. 60 is the default.
        \item {\tt -explicit[=only]} ---  Specifies that explicit columns with control name, alarm status,
                and alarm severity strings be output in addition to the integer
                codes.  If the "only" qualifier is given, the integer codes are
                omitted from the output.
        \item {\tt -verbose} --- Prints out a message when data is taken.
        \item {\tt -precision=\{single|double\}} --- Selects the data type for the statistics columns.
        \item {\tt -updateInterval=<integer>} --- Number of sample sets between each output file update. The default is 1.
        \item {\tt -ezcaTiming[=<timeout>,<retries>]} --- Sets EZCA timeout and retry parameters.
        \item {\tt -noezca} --- Obsolete.
        \item {\tt -dailyFiles} -- The output is sent to the file <SDDSoutputfile>-YYYY-JJJ-MMDD.<N>,
                where YYYY=year, JJJ=Julian day, and MMDD=month+day. A new file is started after 
                the midnight.
        \item {\tt -oncaerror=\{usezero|skiprow|exit\}} --- Selects action taken when a channel access error occurs.
                The default is using zero ({\tt usezero}) for the value of the process variable 
                with the channel access error, and resuming execution. The second option ({\tt skiprow}) is to
    \end{itemize}

\item {\bf see also:}

%
% Insert references to other programs by duplicating this line and 
% replacing <prog> with the program to be referenced:
%

%
% Insert your name and affiliation after the '}'
%
\item {\bf author: M. Borland, ANL} 
\end{itemize}
