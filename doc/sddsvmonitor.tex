% sddsvmonitor: monitor PVs defined by rootname and suffix lists.
\begin{sddsprog}{sddsvmonitor}
\item \textbf{description:}
\verb+sddsvmonitor+ reads values of process variables and writes them to a file at a specified time interval.
This command differs from \verb+sddsmonitor+ in that the monitored
process variables names are specified in two lists.  The first list
gives PV ``rootnames''.  The second list gives suffixes to apply to
each of the rootnames. For each readback step, a page is written to
the output file with the PV rootnames appearing in one column, and the
process variable values in separate data columns for each suffix.

Warning: If the readback values of all of the vector PVs do not
change, then no data sets are written to the output file. This
skipping of duplicate values is intended to keep the size of the
output file as small as possible.  The scalar PVs are not checked for
changes though. In the future an option that allows logging of duplicate
vector PVs may be implemented.

\item \textbf{examples:} 
% 
The pressures readbacks of storage ring vacuum gauges are monitored with the command below.
\begin{verbatim}
sddsvmonitor SRvac.vmon SRvac.vsdds -time=24,hours -interval=1,minute
\end{verbatim}
where the contents of the file \verb+SRvac.vmon+ are
\begin{verbatim}
SDDS1
&parameter name=ListType, type=string &end
&column
 name = "ListData",  type = "string", &end
&data
 mode = "ascii", no_row_counts=1 &end
Rootnames
VM:01:
VM:02:
VM:03:
...
VM:40:

Suffixes
VGC1.PRES
\end{verbatim}
There is only one element in the suffix list of this example. The output file will contain columns
\verb+Rootname+ and \verb+VGC1.PRES+.

\item \textbf{synopsis:} 
\begin{verbatim}
usage: sddsvmonitor [<inputfile> | -rootnames=<file> -suffixes=<file>]
    <outputfile> [{-erase | -append[=recover] | -generations[=digits=<integer>][,delimiter=<string>]}]
    [-steps=<integer> | -time=<value>[,<units>]] [-interval=<value>[,<units>]]
    [-enforceTimeLimit] [-offsetTimeOfDay]
    [-verbose] [-singleShot{=noprompt|stdout}] [-precision={single | double}]
    [-onCAerror={useZero | skipPage | exit}] [-PVlist=<filename>]
    [-scalars=<filename>]
    [-conditions=<filename>,{allMustPass | oneMustPass}[,touchOutput][,retakeStep]]
    [-logDuplicates[=countThreshold=<number>]]
    [-comment=<parameterName>,<text>]
Writes values of process variables to a binary SDDS file.
\end{verbatim}
\item \textbf{files:}
\begin{itemize}
\item \textbf{input file:}\par
The input file is an SDDS file with one data column and one parameter:
\begin{itemize}
        \item {\tt ListData} --- Required string column for the root part or the suffix part 
                of the process variable names.
        \item {\tt ListType} --- Required string parameter for the name
                part type. The only values recognized by sddsvmonitor are ``\verb+rootnames+'' and ``\verb+suffixes+''.
\end{itemize}
The list of process variables is formed by combining all the rootnames and suffixes.

\item \textbf{rootname and suffix files:}\par
An alternative to specifying the rootnames and suffixes with the above input file is
to specify the list of rootnames and suffixes with two separate files, as shown in the usage message
above.
The string data in rootname file must be in column \verb+Rootname+. The string data in suffix file
must be in column \verb+Suffix+. 

\item \textbf{scalar PV input file:}\par
An optional input file for scalar PVs (i.e. regular PVs) can be specified. The required columns are:
\begin{itemize}
        \item {\tt ControlName} --- Required string column for the names of the scalar process variables
                to be monitored.
        \item {\tt ReadbackName} --- Required string column for the names of the parameter in the 
                output file in which the values of the scalar process variables are written.
\end{itemize}


\item \textbf{conditions file:} \par
The conditions file is an optional input specified on the command line which lists
conditions that must be satisfied at each time steps before the data can be logged.

The file is like the main input file, but has numerical columns \verb+LowerLimit+ and \verb+UpperLimit+.
The minimal column set is \verb+ControlName+, which contain the PV names, and the two limits columns above.
Depending on command line options, when any or all PV readback from this file
is outstide the range defined by the corresponding data from \verb+LowerLimit+ and \verb+UpperLimit+,
none of the data of the input file PVs are recorded. 
When this situations occurs for a long period of time, the size of the output file doesn't
grow, and it may appear that the monitoring process has somehow stopped.
It is possible to check the program activity with the \verb+touch+ sub-option
which causes the monitoring program to touch the output file at every step.

\item \textbf{output file:}\par
The output file contains one data column for each suffix named in the input file. By default,
the data type is float (single precision). Other columns are:
\begin{itemize}
        \item {\tt Index} --- Long column for index of rootname.
        \item {\tt Rootname} --- String column for rootnames from the input file.
\end{itemize}
Each reading step produces a new page in the output file.
Time and other miscellaneous parameters are defined: 
\begin{itemize}
        \item {\tt Time} --- Double column for time of readback since the start of epoch. This time data can be used by
        the plotting program {\verb+sddsplot+} to make the best choice of time unit conversions
        for time axis labeling.
        \item {\tt TimeOfDay} --- Float column for system time in units of hours. The time does not wrap around at 24 hours.
        \item {\tt DayOfMonth} --- Float column for system time in units of days. The day does not wrap around at the month boundary.
        \item {\tt Step} --- Long column for step number.
        \item {\tt CAerrors} --- Long column for number of channel access errors at each reading step. 
\end{itemize}

For each scalar PV defined in the \verb+scalars+ command line option a parameter of type double is defined.

Many time-related parameters which don't change values throughout the file are defined:
\begin{itemize}
        \item {\tt TimeStamp} --- String column for time stamp for file.
        \item {\tt PageTimeStamp} --- String column for time stamp for each page. When data
                is appended to an existing file, the new data is written to a new
                page. The {\tt PageTimeStamp} value for the new page is the creation
                date of the new page. The {\tt TimeStamp} value for the new page is the creation 
                date of the very first page.
        \item {\tt StartTime} --- Double column for start time from {\tt C} time call cast to type double.
        \item {\tt YearStartTime} --- Double column for start time of present year from {\tt C} 
                time call cast to type double.
        \item {\verb+StartYear+} --- Short parameter for the year when the file was started.
        \item {\verb+StartJulianDay+} --- Short parameter for the day when the file was started.
        \item {\verb+StartMonth+} --- Short parameter for the month when the file was started.
        \item {\verb+StartDayOfMonth+} --- Short parameter for the day of month when the file was started.
        \item {\verb+StartHour+} --- Short parameter for the hour when the file was started.
\end{itemize}
\end{itemize}

\item \textbf{switches:}
    \begin{itemize}
%   \item {\tt -pipe[=input][,output]} --- The standard SDDS Toolkit pipe option.
        \item {\tt -rootnames=<file>} --- Specifies input file for rootnames. String values
                must be in column \verb+Rootname+.
        \item {\tt -suffixes=<file>} --- Specifies input file for suffixes. String values
                must be in column \verb+Suffix+.
        \item {\tt -scalars=<filename>} ---  Specifies input file for scalar PV names. The values
                are logged as parameters.
        \item {\verb+-conditions=<filename>,{allMustPass | oneMustPass}[,touchOutput][,retakeStep]+} ---
                Names an SDDS file containing PVs to read and limits on each PV that must
                be satisfied for data to be taken and logged.  The file is like the main
                input file, but has numerical columns \verb+LowerLimit+ and \verb+UpperLimit+.

                One of \verb+allMustPass+ or \verb+oneMustPass+ must be specified. It would make sense
                to use \verb+allMustPass+ in most monitoring applications.
                If \verb+touchOutput+ is present, then the output file is touched, even if no data
                is written. This way, one can determine by the time stamp of the file
                whether the monitoring job is still alive
                when the conditions fail for a long period of time. If \verb+retakeStep+ is
                present, then the value of \verb+Step+ in the output file is not
                incremented until the conditions pass, and data is written to the output file.

        \item {\tt -erase} --- If the output file already exists, the it will be overwritten
                by sddsvmonitor.
        \item {\tt -append[=recover]} --- If the output file already exists, then append the new readings.
                The output file must have previously been generated by \verb+sddsvmonitor+ using the same
                information in the input files.
                The {\verb+recover+} option allows an attempt
                to recover the data using \verb+sddsconvert+ if the input file is somehow corrupted.
        \item {\verb+-generations[=digits=<integer>][,delimiter=<string>]+} ---
                The output is sent to the file \verb+<SDDSoutputfile>-<N>+, where \verb+<N>+ is
                   the smallest positive integer such that the file does not already
                   exist.   By default, four digits are used for formatting \verb+<N>+, so that
                   the first generation number is 0001.
        \item {\tt -interval=<real-value>[,<time-units>]} --- Specifies the interval between readings. The time
                interval is implemented with a call to usleep between calls to the control system.
                Because the calls to the control system make take up a significant amount of time, the average
                effective time interval may sometimes be longer specified. 
        \item {\tt -steps=<integer-value>} --- Number of readbacks for each process variable before normal exiting.
        \item {\tt -time=<real-value>[,<time-units>]} --- Total time for monitoring. Valid time units are
                seconds, minutes, hours, and days. The program calculates the number of steps by dividing this time
                by the interval. The completion time may be longer, because the time interval in not guaranteed.
        \item {\tt -enforceTimeLimit} --- Enforces the total monitoring time given by \verb+-time+ even if
                the expected number of steps has not been taken.
        \item {\tt -offsetTimeOfDay} --- Adjusts the starting \verb+TimeOfDay+ value so that it corresponds
                to the day on which most data are taken.
        \item {\tt -verbose} --- Prints out a message when data is taken.
        \item {\verb+-singleShot{=noprompt|stdout}+} --- A single read is prompted at the terminal
                and initiated by a \verb+<cr>+ key press. The time interval is disabled.
                With \verb+noprompt+ present, no prompt is written to the terminal, but a \verb+<cr>+
                is still expected. Using \verb+stdout+ directs the prompt to standard output.
                Typing ``q'' or ``Q'' terminates the monitoring.
        \item {\tt -precision=\{single|double\}} --- Sets the data precision for PV values in the output file.
        \item {\tt -oncaerror=\{useZero|skipPage|exit\}} --- Selects action taken when a channel access error occurs.
                The default is using zero (\verb+useZero+) for the value of the process variable
                with the channel access error, and resuming execution. The second option (\verb+skipPage+) is to
                simply throw away all the data for that read step, and resume execution.
                The third option is to exit the program.
        \item {\tt -PVlist=<filename>} --- Specifies a file in which to write
                the names of all PVs monitored.
        \item {\verb+-logDuplicates[=countThreshold=<number>]+} --- Specifies that data should be
                logged even if it is exactly the same as the last data.
        \item {\verb+-comment=<parameterName>,<text>+} ---
                Gives the parameter name for a comment to be placed in the SDDS output file,
                along with the text to be placed in the file.
    \end{itemize}

\item \textbf{see also:}
    \begin{itemize}
% replacing <prog> with the program to be referenced:
    \item \progref{sddsvmonitor}
    \item \progref{sddswmonitor}
    \item \progref{sddssnapshot}
    \end{itemize}
\item \textbf{author:} M. Borland and L. Emery, ANL 
\end{sddsprog}
