%
% cavget documentation.
%
\begin{sddsprog}{cavget}
\item \textbf{description:}
\verb+cavget+ retrieves values of one or more EPICS process variables using Channel Access or PVAccess.
It supports formatting, repeated acquisition and statistical analysis for scripting and data collection.
\item \textbf{examples:}
\begin{flushleft}{\tt
cavget -list=my:pv\\
cavget -repeat=number=10,pause=1 -labeled -list=quadrupole:current\\
cavget -statistics=number=100,pause=0.1,format=SDDS,file=stats.sdds -list=cavit:readback
}\end{flushleft}
\item \textbf{synopsis:}
\begin{flushleft}{\tt
usage: cavget [-list=<string>[=<value>][,<string>[=<value>]...]]\
[-range=begin=<integer>,end=<integer>[,format=<string>][,interval=<integer>]]\
[-floatformat=<printfString>] [-charArray] [-delimiter=<string>] [-labeled]\
[-noQuotes] [-embrace=start=<string>,end=<string>] [-cavputForm]\
[-statistics=number=<value>,pause=<value>[,format=[tagvalue][pretty][SDDS,file=<filename>]]]\
[-pendIoTime=<seconds>] [-dryRun] [-repeat=number=<integer>,pause=<seconds>[,average[,sigma]]]\
[-numerical] [-errorValue=<string>] [-excludeErrors]\
[-despike[[neighbors=<integer>][,passes=<integer>][,averageOf=<integer>][,threshold=<value>]]]\
[-provider={ca|pva}] [-info] [-printErrors] pvname [...]
}\end{flushleft}
\item \textbf{switches:}
\begin{itemize}
  \item {\tt -list} --- Build PV names from components or explicit names.
  \item {\tt -range} --- Append integer ranges to PV prefixes.
  \item {\tt -floatformat} --- Set printf-style format for floating point numbers.
  \item {\tt -charArray} --- Print character arrays as strings.
  \item {\tt -delimiter} --- String used to separate values.
  \item {\tt -labeled} --- Prepend PV names to values.
  \item {\tt -noQuotes} --- Suppress quotes around string values.
  \item {\tt -embrace} --- Surround output with given start and end strings.
  \item {\tt -cavputForm} --- Format output suitable for input to \verb+cavput+.
  \item {\tt -statistics} --- Compute statistics over repeated readings.
  \item {\tt -pendIoTime} --- Maximum time to wait for connections.
  \item {\tt -dryRun} --- List PV names without fetching values.
  \item {\tt -repeat} --- Take multiple readings with optional pause.
  \item {\tt -numerical} --- Use numerical values for enumerations.
  \item {\tt -errorValue} --- Substitute a value if a read fails.
  \item {\tt -excludeErrors} --- Omit values if a read fails.
  \item {\tt -despike} --- Remove outliers using neighbors/passes settings.
  \item {\tt -provider} --- Select Channel Access (ca) or PVAccess (pva).
  \item {\tt -info} --- Show connection and data type information.
  \item {\tt -printErrors} --- Print errors without aborting.
\end{itemize}
\item \textbf{see also:}
\begin{itemize}
  \item \progref{cavput}
  \item \progref{cawait}
\end{itemize}
\item \textbf{author:} Michael Borland, Robert Soliday, ANL/APS.
\end{sddsprog}
