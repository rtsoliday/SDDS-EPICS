%
% Template for making SDDS Toolkit manual entries.
%
\begin{latexonly}
\newpage
\end{latexonly}

%
% Substitute the program name for sddssnapshot
%
\subsection{sddssnapshot}
\label{sddssnapshot}

\begin{itemize}
\item {\bf description:}
%
% Insert text of description (typicall a paragraph) here.
%
\verb+sddssnapshot+ reads values of process variables and writes them to a file.
An input file lists the process variables to be read.

\verb+sddssnapshot+ differs from \verb+burtrb+ in that \verb+sddssnapshot+  may operate in a server mode
in which a new file is written to the named output file whenever the signal SIGUSR1 is received 
by \verb+sddssnapshot+.
Another improvement over \verb+burtrb+ is that
all data in the input file (even those not needed by the program) are transfered to the output file.
\verb+sddssnapshot+ is more for data collection as opposed to backup and restore.

\item {\bf example:} 
%
% Insert text of examples in this section.  Examples should be simple and
% should be proceeded by a brief description.  Wrap the commands for each
% example in the following construct:
% 
%
The state of the APS storage ring is saved by writing 
values of process variables listed in SR.req
to the snapshot file SR.snp:
\begin{verbatim}
sddssnapshot SR.req SR.snp -nameOfData="Value"
\end{verbatim}
where the contents of the file \verb+SR.req+ are
\begin{verbatim}
SDDS1
&column
 name = ControlName,  type = string, &end
&data
 mode = ascii, &end
&data
 mode = "ascii", no_row_counts=1 &end
S1A:Q1:CurrentAO
S1A:Q2:CurrentAO
...
\end{verbatim}

\item {\bf synopsis:} 
%
% Insert usage message here:
%
\begin{verbatim}
usage: sddssnapshot [-pipe[=input][,output]] [<input>] [<output>]
[-ezcaTiming=<timeout>,<retries>] [-unitsOfData=<string>] [-nameOfData=<string>]
[-serverMode=<pidFile>] [-average=<number>,<intervalInSec>]
Takes a snapshot of EPICS scalar process variables.
Requires the column "ControlName" with the process variable names.
For server mode, writes a new file to the given filename whenever
SIGUSR1 is received.  Exits when SIGUSR2 is received.
\end{verbatim}
\item {\bf files:}
% Describe the files that are used and produced
\begin{itemize}
\item {\bf input file:}\par
The input file is an SDDS file with at least one column:
\begin{itemize}
        \item {\tt ControlName} --- Required string column for the process variable or device name.
\end{itemize}

\item {\bf output file:}\par
The output file contains all columns of the input file including those
not needed by the program plus a column named on the command line option {\tt -nameOfData}.
This column is defined as a double type and contains the readback values. 
Optionally the units of that readback column may be specified on the command line.
Of course this option is useful only if all the process variables have the same units, as in
the case of recording orbit values from all bpms.
If the {\tt -average} option is requested, then an additional double column is created with the name
of the readback column with the ``StDev'' appended to it.

\item {\bf pid file:}\par
A process id file is created with option {\verb+serverMode=<pidFile>+}. This file contains a single
number which is the pid number of the running {\verb+sddssnapshot+} process.
\end{itemize}
%
\item {\bf switches:}
%
% Describe the switches that are available
%
    \begin{itemize}
%
%   \item {\tt -pipe[=input][,output]} --- The standard SDDS Toolkit pipe option.
%
        \item {\tt -pipe[=input][,output]} --- The standard SDDS Toolkit pipe option.
        \item {\tt -ezcaTiming=<timeout>,<retries>} --- Specifies tuning of ezca, the channel access 
        interface used by \verb+sddssnapshot+.
        \item {\tt -nameOfData=<string>} --- Column name to be given to the data collected. Default name is ``Value''.
        \item {\tt -unitsOfData=<string>} --- Optional. Name given to the units field of the column definition 
                        of the data to be collected. Default value is the null string.
        \item {\tt -serverMode=<pidFile>} --- Optional. Enables the server more. The file specified will be created and contain
                the process number of the present \verb+sddssnapshot+ process. This file is the mechanism
                through which the user will know to which process should the SIGUSR1 be sent.
                To activate one snapshot write, the user can type
                the command ``\verb+kill -SIGUSR1 `cat <pidFile>+`''.
        \item {\tt -average=<number>,<intervalInSec> } --- Optional. On can specify the number of readings to
                average and the number of seconds interval between readings.
    \end{itemize}

\item {\bf see also:}
    \begin{itemize}
%
% Insert references to other programs by duplicating this line and 
% replacing <prog> with the program to be referenced:
%
    \item \progref{burtrb}
    \item \progref{sddsmonitor}
    \end{itemize}
%
% Insert your name and affiliation after the '}'
%
\item {\bf author: M. Borland, ANL} 
\end{itemize}
