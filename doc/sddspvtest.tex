%
% Template for making SDDS Toolkit manual entries.
%
\begin{latexonly}
\newpage
\end{latexonly}

%
% Substitute the program name for sddsmonitor
%
\subsection{sddspvtest}
\label{sddspvtest}

\begin{itemize}
\item {\bf description:}
%
% Insert text of description (typicall a paragraph) here.
%
\verb+sddspvtest+ tests the process variable given by the inputfile are out-of-range
or not and sets the number of out-of-range process variables to a control PV.   

\item {\bf example:} 
%
% Insert text of examples in this section.  Examples should be simple and
% should be proceeded by a brief description.  Wrap the commands for each
% example in the following construct:
% 
%
\begin{verbatim}
sddspvtest linac.sdds -time=20 -runControlPV={string=shang:ControlLawRC,pingTimeout=4}\
  -runControlDescription=string=hi
\end{verbatim}
where the contents of the file \verb+linac.sdds+ are
\begin{verbatim}
SDDS1
&description text="Namecapture BURT Request File", contents="BURT Request", &end
&column name=ControlName, type=string,  &end
&column name=MaximumValue, type=double,  &end
&column name=MinimumValue, type=double,  &end
&data mode=ascii, &end
! page number 1
linac
                  10
soliday:PM1:X:positionM  1.000000000000000e+00 -1.000000000000000e+00 
soliday:PM1:Y:positionM  1.000000000000000e+00 -1.000000000000000e+00 
soliday:PM2:X:positionM  1.000000000000000e+00 -1.000000000000000e+00 
soliday:PM2:Y:positionM  1.000000000000000e+00 -1.000000000000000e+00 
soliday:PM3:X:positionM  1.000000000000000e+00 -1.000000000000000e+00 
soliday:PM3:Y:positionM  1.000000000000000e+00 -1.000000000000000e+00 
soliday:PM4:X:positionM  1.000000000000000e+00 -1.000000000000000e+00 
soliday:PM4:Y:positionM  1.000000000000000e+00 -1.000000000000000e+00 
soliday:PM5:X:positionM  1.000000000000000e+00 -1.000000000000000e+00 
soliday:PM5:Y:positionM  1.000000000000000e+00 -1.000000000000000e+00 
.......

\end{verbatim}

\item {\bf synopsis:} 
%
% Insert usage message here:
%
\begin{verbatim}
usage: sddspvtest <inputFile> [-pvOutput=<pvName>] 
  [-time=<timeToRun>,<timeUnits>] [-interval=<timeInterval>,<timeUnits>] 
  [-runControlPV={string=<string>|parameter=<string>},pingTimeout=<value>,
  pingInterval=<value>] 
  [-runControlDescription={string=<string>|parameter=<string>}] 
  [-pendIOtime=<value>] [-verbose] [-testValues=<file>[,limit=<number>]]
\end{verbatim}
\item {\bf files:}
% Describe the files that are used and produced
\begin{itemize}
\item {\bf input file:} \par
The variable input file is an SDDS file with one string column: ControlName, which is required
and gives the list of control correctors (process variables or knobs). It also contains two double
double columns: MaximumValue and MinimumValue, which are also required.
seconds between two settings of the correctors.
\end{itemize}

%
\item {\bf switches:}
%
% Describe the switches that are available
%
    \begin{itemize}
%
%   \item {\tt -pipe[=input][,output]} --- The standard SDDS Toolkit pipe option.
%
        \item {\tt -pvOutput} --- optional. the output pv name for storing the testing results
               of PVs in the input. If it is not given,the results are printed out.
        \item {\tt -time} ---required. Total time for testing process variable.
               Valid time units are seconds,minutes,hours, or days. 
         \item {\tt -interval} ---optional. Desired time interval for testing, the units are same as time.
        \item {\tt -runControlPV} --- specifies the runControl PV record. string|parameter is required,
                pingInterval and pingTimeout are optional.
        \item {\tt -runControlDescription} ---specifies a string parameter whose value is a runControl
                PV description record.
        \item {\tt -verbose} -- print out messages. 
        \item {\tt -pendIOtime} -- sets the maximum time to wait for return of each value.
        \item {\tt -testValues<file>,[limit=<number>]} --
           file is sdds format file containing minimum and maximum values of PV's specifying
           a range outside of which the feedback is temporarily suspended. Column names are
           ControlName, MinimumValue, MaximumValue. Optional column names are SleepTime, 
           ResetTime. limits is the maixum number of failure times. The program will be
           terminated when the continuous failure times reaches the limit.
    \end{itemize}
\item {\bf see also:}
    \begin{itemize}
%
% Insert references to other programs by duplicating this line and 
% replacing <prog> with the program to be referenced:
%
    \item \progref{sddscontrollaw}
    \end{itemize}
%
% Insert your name and affiliation after the '}'
%
\item {\bf author: H. Shang, ANL}
\end{itemize}
