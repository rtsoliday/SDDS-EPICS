%
% Template for making SDDS Toolkit manual entries.
%
\begin{latexonly}
\newpage
\end{latexonly}

%
% Substitute the program name for sddsmonitor
%
\subsection{sddswmonitor}
\label{sddswmonitor}

\begin{itemize}
\item {\bf description:}
%
% Insert text of description (typicall a paragraph) here.
%
\verb+sddswmonitor+ reads values of waveform process variables 
and writes them to a file at a specified time interval.
An input file defines the process variables to be monitored.

Warning: If the readback values of all of the waveform PVs do not
change, then no data sets are written to the output file. This
skipping of duplicate values is intended to keep the size of the
output file as small as possible.  The scalar PVs are not checked for
changes though. In the future an option that allows logging of duplicate
waveform PVs may be implemented.

\item {\bf example:} 
%
% Insert text of examples in this section.  Examples should be simple and
% should be proceeded by a brief description.  Wrap the commands for each
% example in the following construct:
% 
%
The history of a beam position monitor readback is collected with this command:
\begin{verbatim}
sddswmonitor SlowBh.wmon SlowBh.sdds -step=1
\end{verbatim}
where the contents of the file \verb+SlowBh.wmon+ are
\begin{verbatim}
SDDS1
&description &end
&description
 contents = "sddssequence output", &end
&parameter
 name = WaveformLength, type=long, &end
&column
 name = WaveformPV,  type = string, &end
&column
 name = WaveformName,  type = string, &end
&data
 mode = ascii, &end
! page number 1
512             ! WaveformLength
2               ! number of rows
S1A:P1:bh:x_wf S1A:P1:x
S1A:P1:bh:y_wf S1A:P1:y
\end{verbatim}
\item {\bf synopsis:} 
%
% Insert usage message here:
%
\begin{verbatim}
usage: sddswmonitor {<inputfile> | -PVnames=<name>[,<name>]} <outputfile>
    [{-erase | -generations[=digits=<integer>][,delimiter=<string>]}]
    [-steps=<integer> | -time=<value>[,<units>]] [-interval=<value>[,<units>]]
    [-verbose] [-singleShot[=noprompt]] [-precision={single | double}]
    [-onCAerror={useZero | skipPage | exit}] 
    [-scalars=<filename>]
    [-conditions=<filename>,{allMustPass | oneMustPass}[,touchOutput][,retakeStep]]
    [-ezcaTime=<timeout>,<retries>] 
    [-comment=<parameterName>,<text>]
Writes values of waveform process variables to a binary SDDS file.
\end{verbatim}
\item {\bf files:}
% Describe the files that are used and produced
\begin{itemize}
\item {\bf input file:}\par
The input file is an SDDS file with two required columns and one required parameter:
\begin{itemize}
        \item {\verb+WaveformLength+} --- Required long parameter for the length of the waveform PV's. All
                WaveformPVs are expected to have this length. 
        \item {\verb+WaveformPV+}  --- Required string column for the names of the waveform process variables.
        \item {\verb+WaveformName+} --- Required string column for the names of the data columns in the output file.
\end{itemize}

\item {\bf scalar PV input file:}\par
An optional input file for scalar PVs (i.e. regular PVs) can be specified. The required columns are:
\begin{itemize}
        \item {\tt ControlName} --- Required string column for the names of the scalar process variables
                to be monitored.
        \item {\tt ReadbackName} --- Required string column for the names of the parameter in the 
                output file in which the values of the scalar process variables are written.
\end{itemize}

\item {\bf conditions file:} \par
The conditions file is an optional input file specified on the command line which lists
conditions that must be satisfied at each time step before the data can be logged.

The file is like the main input file, but has numerical columns \verb+LowerLimit+ and \verb+UpperLimit+.
The minimal column set is \verb+ControlName+, which contain the PV names, and the two limits columns above.
Depending on comand line options, when any or all PV readback from this file
is outstide the range defined by the corresponding data from \verb+LowerLimit+ and \verb+UpperLimit+,
none of the data of the input file PVs are recorded. 
When this situations occurs for a long period of time, the size of the output file doesn't
grow, and it may appear that the monitoring process has somehow stopped.
It is possible to check the program activity with the \verb+touch+ sub-option
which causes the monitoring program to touch the output file at every step.

\item {\bf output file:}\par
The output file contains one data column for each waveform process variable named in the input file. The names
of the data columns are given by the values of {\verb+WaveformName+} in the input file. The units are obtained
internally from the EPICS database. An additional long column \verb+Index+ is created that give the index
of each point in the waveform.

The values of the scalar PVs are written to parameters with names given by the {\tt ReadbackName} column
of the optional scalars input file. The units are obtained internally from the EPICS database. Scalar
PVs that fail to connect when the program starts are omitted from the output file.

By default, the data type is float (single precision). Each reading step produces a new page in the output file.

Time and other miscellaneous parameters are defined: 
\begin{itemize}
        \item {\tt Time} --- Double column for elapsed time of readback since the start of epoch.
        \item {\tt TimeOfDay} --- Float column for system time in units of hours. The time does not wrap around at 24 hours.
        \item {\tt DayOfMonth} --- Float column for system time in units of days. The day does not wrap around at the month boundary.
        \item {\tt Step} --- Long column for step number.
        \item {\tt CAerrors} --- Long column for number of channel access errors at each reading step. 
\end{itemize}

For each connected scalar PV defined in the \verb+scalars+ command line option a parameter of type double is defined.

Many additional parameters which don't change values throughout the file are defined:
\begin{itemize}
        \item {\tt TimeStamp} --- String column for time stamp for file.
        \item {\tt PageTimeStamp} --- String column for time stamp for each page.
        \item {\tt StartTime} --- Double column for start time from {\tt C} time call cast to type double.
        \item {\tt YearStartTime} --- Double column for start time of present year from {\tt C}
                time call cast to type double.
        \item {\verb+StartYear+} --- Short parameter for the year when the file was started.
        \item {\verb+StartJulianDay+} --- Short parameter for the day when the file was started.
        \item {\verb+StartMonth+} --- Short parameter for the month when the file was started.
        \item {\verb+StartDayOfMonth+} --- Short parameter for the day of month when the file was started.
        \item {\verb+StartHour+} --- Short parameter for the hour when the file was started.
\end{itemize}
\end{itemize}

%
\item {\bf switches:}
%
% Describe the switches that are available
%
    \begin{itemize}
%
%   \item {\tt -pipe[=input][,output]} --- The standard SDDS Toolkit pipe option.
%
        \item {\verb+-PVnames=<name>[,<name>]+} ---
                   specifies a list of PV names to read.  It the waveforms are
                   of different lengths, the short ones are padded with zeros.
        \item {\tt -scalars=<filename>} ---  Specifies input file for scalar PV names. The values
                are logged as parameters.
        \item {\tt -erase} --- If the output file already exists, then it will be overwritten
                by \verb+sddswmonitor+.
        \item {\verb+-generations[=digits=<integer>][,delimiter=<string>]+} ---
                The output is sent to the file \verb+<SDDSoutputfile>-<N>+, where \verb+<N>+ is
                   the smallest positive integer such that the file does not already
                   exist.   By default, four digits are used for formating \verb+<N>+, so that
                   the first generation number is 0001.
        \item {\tt -interval=<real-value>[,<time-units>]} --- Specifies the interval between readings. The time
                interval is implemented with a call to usleep between calls to the control system.
                because the calls to the control system may take up a significant amount of time, the average
                effective time interval may be longer than specified. 
        \item {\tt -steps=<integer-value>} --- Number of readbacks for each process variable before normal exiting.
        \item {\tt -time=<real-value>[,<time-units>]} --- Total time for monitoring. Valid time units are
                seconds, minutes, hours, and days. The program calculates the number of steps by dividing this time
                by the interval. The completion time may be longer, because the time interval in not garanteed.
        \item {\tt -verbose} --- prints out a message when data is taken.
        \item {\verb+-singleShot[=noprompt]+} --- a single read is prompted at the terminal
                and initiated by a \verb+<cr>+ key press. The time interval is disabled. 
                With \verb+noprompt+ present, no prompt is written to the terminal, but a \verb+<cr>+
                is still expected. Typing ``q'' or ``Q'' terminates the monitoring.
        \item {\tt -oncaerror={usezero|skiprow|exit}} --- Selects action taken when a channel access error ocurrs.
                The default is using zero ({\tt usezero}) for the value of the process variable 
                with the channel access error, and resuming execution. The second option ({\tt skiprow}) is to
                simply throw away all the data for that read step, and resume execution.
                the third option is to exit the program.
        \item {\tt -ezcaTiming[=<timeout>,<retries>]} --- Sets EZCA timeout and retry parameters.
        \item {\tt -scalars=<filename>} --- Specifies sddsmonitor input file to get names of scalar PVs
                   from.  These will be logged in the output file as parameters.
        \item {\verb+-conditions=<filename>,{allMustPass | oneMustPass}[,touchOutput][,retakeStep]]+} --- 
                Names an SDDS file containing PVs to read and limits on each PV that must
                be satisfied for data to be taken and logged.  The file is like the main
                input file, but has numerical columns LowerLimit and UpperLimit.
                
                One of \verb+allMustPass+ or \verb+oneMustPass+ must be specified. It would make sense
                to use \verb+allMustPass+ in most monitoring applications.
                If \verb+touchOutput+ is present, then the output file is touched, even if no data
                is written. This way, one can determine by the time stamp of the file
                whether the monitoring job is still alive
                when the conditions fail for a long period of time. If \verb+retakeStep+ is
                present, then the value of \verb+Step+ in the output file is not
                incremented until the conditions pass, and data is written to the output file.
        \item {\verb+-comment=<parameterName>,<text>+} ---
                Gives the parameter name for a comment to be placed in the SDDS output file,
                along with the text to be placed in the file.
    \end{itemize}

\item {\bf see also:}
    \begin{itemize}
%
% Insert references to other programs by duplicating this line and 
% replacing <prog> with the program to be referenced:
%
    \item \progref{sddsmonitor}
    \item \progref{sddsvmonitor}
    \item \progref{sddssnapshot}
    \end{itemize}
%
% Insert your name and affiliation after the '}'
%
\item {\bf author: M. Borland and L. Emery, ANL} 
\end{itemize}
