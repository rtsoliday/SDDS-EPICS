% Template for making SDDS Toolkit manual entries.
\begin{latexonly}
\newpage
\end{latexonly}

\subsection{sddswget}
\label{sddswget}

\begin{itemize}
\item {\bf description:}
  \verb+sddswget+ reads waveform process variables and writes the values to a binary SDDS file.  Waveform PV names may come from an SDDS input file or from the command line.  Each output page contains a single waveform with time stamps.

\item {\bf examples:}
  The following command fetches two waveform PVs and stores them in an SDDS file:
  \begin{flushleft}{\tt
  sddswget -PVnames=rf:wave1,rf:wave2 waveforms.sdds
  }\end{flushleft}

\item {\bf synopsis:}
  \begin{flushleft}{\tt
  sddswget\\
      [<inputfile> [-inputStyle={wget|wmonitor|automatic}] | -PVnames=<name>[,<name>]]\\
      <outputfile> [-pendIOtime=<seconds>] [-pipe[=input][,output]]\\
      [-provider={ca|pva}]\\
  }\end{flushleft}

\item {\bf files:}
  The output file contains parameters \verb+Time+, \verb+TimeStamp+, and \verb+WaveformPV+ along with columns \verb+Index+ and the waveform data.

\item {\bf switches:}
  \begin{itemize}
    \item {\tt -PVnames=<name>[,<name>]} --- list waveform PVs on the command line.
    \item {\tt -inputStyle={wget|wmonitor|automatic}} --- choose how PV names are read from the input file.  Wmonitor style expects a \verb+WaveformPV+ column; wget style uses a \verb+WaveformPV+ parameter.
    \item {\tt -pendIOtime=<seconds>} --- EPICS connection timeout.
    \item {\tt -pipe[=input][,output]} --- standard SDDS Toolkit pipe option.
    \item {\tt -provider={ca|pva}} --- select EPICS Channel Access (default) or PVAccess.
  \end{itemize}

\item {\bf see also:}
  \begin{itemize}
    \item \progref{sddswmonitor}
    \item \progref{sddswput}
  \end{itemize}

\item {\bf author: M. Borland, ANL/APS}
\end{itemize}
