% sdds2dfeedforward: apply two-dimensional feedforward tables to actuators.
\begin{sddsprog}{sdds2dfeedforward}
\item \textbf{description:}
\verb+sdds2dfeedforward+ performs two-dimensional feedforward on process variables.
It sets one or more actuators according to a predetermined function of two readback
values.  The user supplies tables giving the actuator value as a function of two
readbacks.  Each table is provided as an SDDS page containing the columns
\verb+Readback1Value+, \verb+Readback2Value+, and \verb+ActuatorValue+.  Parameters
\verb+Readback1Name+, \verb+Readback2Name+, and \verb+ActuatorName+ specify the
corresponding process variables.  Optional parameters include
\verb+ReadbackChangeThreshold+ and \verb+ActuatorChangeLimit+.

\item \textbf{examples:}
A basic invocation reads an SDDS table and applies it every ten seconds:
\begin{flushleft}{\tt
sdds2dfeedforward ID1Compensation.sdds -interval=10
}\end{flushleft}
where \verb+ID1Compensation.sdds+ might contain
\begin{flushleft}{\tt
Readback1Name = ID1:X \quad Readback2Name = ID1:Y \quad ActuatorName = S1B:H1:CurrentAO\\
 Readback1Value  Readback2Value  ActuatorValue\\
------------------------------\\
       150.000       0.000          0.000\\
       100.000      50.000          1.000\\
        50.000     100.000         10.000\\
}\end{flushleft}
\item \textbf{synopsis:}
\begin{flushleft}{\tt
sdds2dfeedforward <inputfile>\
       [-controlLog=<rootname>]\
       [-interval=<real-value>] [-steps=<integer-value>]\
       [-verbose] [-dryRun] [-offsetOnly]\
       [-averageOf=<number>[,interval=<seconds>]]\
       [-advance=<seconds>]\
       [-testValues=<SDDSfile>]\
       [-runControlPV=string=<string>[,pingTimeout=<value>]]\
       [-runControlDescription=<string>]\
       [-CASecurityTest]\
       [-pendIOtime=<seconds>] [-infiniteLoop]
}\end{flushleft}
\item \textbf{files:}
\begin{itemize}
  \item \textbf{input file:} An SDDS file with parameters \verb+Readback1Name+,
  \verb+Readback2Name+, and \verb+ActuatorName+.  Optional parameters include
  \verb+ReadbackChangeThreshold+ and \verb+ActuatorChangeLimit+.  Required
  columns are \verb+Readback1Value+, \verb+Readback2Value+, and
  \verb+ActuatorValue+.
\end{itemize}
\item \textbf{switches:}
\begin{itemize}
  \item {\tt interval=<real-value>} --- Interval in seconds between corrections.
  \item {\tt steps=<integer-value>} --- Number of iterations before exiting.
  \item {\tt verbose} --- Print additional information.
  \item {\tt dryRun} --- Readbacks are read but actuators are not changed.
  \item {\tt offsetOnly} --- Offset tables using initial readback values so
               no change is made until a readback moves.
  \item {\tt averageOf=<number>[,interval=<seconds>]} --- Average multiple
               readings of each readback before computing corrections.
  \item {\tt advance=<seconds>} --- Advance readback values based on their
               ramp rate by the given number of seconds.
  \item {\tt testValues=<file>} --- SDDS file giving limits for PV tests that
               suspend feedforward when violated.
  \item {\tt runControlPV=string=<string>[,pingTimeout=<value>]} --- Use a
               run-control PV and optional ping timeout.
  \item {\tt runControlDescription=<string>} --- Description for the
               run-control record.
  \item {\tt CASecurityTest} --- Check Channel Access security before
               writing values.
  \item {\tt pendIOtime=<seconds>} --- Time to wait for Channel Access
               operations.
  \item {\tt infiniteLoop} --- Run indefinitely.
  \item {\tt controlLog=<rootname>} --- Log input and output data to an
               SDDS file using the given root name.
\end{itemize}
\item \textbf{see also:}
\begin{itemize}
  \item \progref{sddsfeedforward}
\end{itemize}
\item \textbf{author:} M. Borland, ANL
\end{sddsprog}
