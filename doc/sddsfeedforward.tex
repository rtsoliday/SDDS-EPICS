% sddsfeedforward: apply feedforward tables to drive actuators from readbacks.
\begin{sddsprog}{sddsfeedforward}
\item {\bf description:}
\verb+sddsfeedforward+ performs feedforward on process variables.
Feedforward essentially means that one or more process variables are
set according to a predetermined function of other process variables.
In \verb+sddsfeedforward+, the user supplies a series of tables that
are used to interpolate values of a single actuator as a function of a
single readback.  Any number of tables may be given, so that any
number of actuators may be driven by any number of readbacks.  Each
actuator is, however, driven by only one readback.  One readback may
drive any number of actuators.

\item {\bf example:} 
% 
\begin{verbatim}
sddsfeedforward ID1GapCompensation.sdds -interval=10
\end{verbatim}
where the file ID1GapCompensation.sdds contains the following
data:
\begin{verbatim}
ReadbackName = ID1:Gap             ActuatorName = S1B:H1:CurrentAO  
 ReadbackValue  ActuatorValue 
------------------------------
       150.000          0.000 
       100.000          0.000 
        50.000          1.000 
        40.000         10.000 
        30.000         20.000 
        20.000         35.000 
        15.000         74.000 
        11.000         99.000 
\end{verbatim}
At each iteration step, ID1:Gap is read and \verb+sddsfeedforward+ interpolates a new value for
\verb+S1B:H1:CurrentAO+ based on this table, which it writes to the PV.
\item {\bf synopsis:} 
\begin{verbatim}
sddsfeedforward <inputfile>
       [-controlLog=<rootname>]
       [-interval=<real-value>] [-steps=<integer-value>]
       [-verbose] [-dryRun] [-offsetOnly]
       [-averageOf=<number>[,interval=<seconds>]]
       [-advance=<seconds>]
       [-testValues=<SDDSfile>]
       [-runControlPV=string=<string>[,pingTimeout=<value>]]
       [-runControlDescription=<string>] [-CASecurityTest]
       [-pendIOtime=<seconds>] [-infiniteLoop]
       [-ezcaTiming=<timeout>,<retries>]
\end{verbatim}
\item {\bf files:}
\begin{itemize}
\item {\bf input file:} \par
The input file is an SDDS file with the following elements:
\begin{itemize}
\item Parameters
\begin{itemize}
\item {\tt ReadbackName} --- Required string parameter giving the name of the
        readback for this table.  This is the process variable that is read from
        EPICS.  Its value is used to interpolate a new value for the actuator.
\item {\tt ActuatorName} --- Required string parameter giving the name of the
        actuator for this table.  This is the process variable that is set by
        the program according to interpolation with the readback values.
\item {\tt ReadbackChangeThreshold} --- Optional floating-point value giving
        the amount by which the process variable named in {\tt ReadbackName}
        must change from the previous iteration's value in order for a
        change to be made for the actuator.  Most useful if the readback is
        a little noisy and one doesn't want the actuator to be changed
        pointlessly.  If the same process variable is named as readback for
        several actuators, then the smallest value of the change threshold
        is the one used.
\item {\tt ActuatorChangeLimit} --- Optional floating-point value giving 
        the maximum amount by which the actuator should be changed in a
        single iteration.  If several actuators have the same
        readback process variable, then they are limited together by the
        most constraining of the limits.  In such a case, the vector of
        changes is scaled to bring all the changes within the respective
        limits.
\end{itemize}
\item Columns
\begin{itemize}
\item {\tt ReadbackValue} --- Required floating-point column giving values
        of the readback for an interpolation table.
\item {\tt ActuatorValue} --- Required floating-point column giving values
        of the actuator for an interpolation table.
\end{itemize}
\end{itemize}

\item {\bf test values file:} \par
To make \verb+sddsfeedforward+ more robust, one can implement tests on any process variable,
not necessarily those involved in the feedforward itself. If any of the tests fail, then the 
feedforward is suspended until the test succeeds. The test consist of checking whether a process 
variable is within a specified range or not. The \verb+testValues+ file has three required columns and
one optional one:
\begin{itemize}
        \item {\tt ControlName} --- Required string column giving process variables names to test.
        \item {\tt MinimumValue}, {\tt MaximumValue} --- Required floating-point columns, defining
                a valid range for corresponding PVs.  It is an error if
                {\tt MinimumValue} $>$ {\tt MaximumValue}.
        \item {\tt SleepTime} --- Optional floating-point column, specifying sleep (or pause) time before
                attempting another test.
\end{itemize}
\end{itemize}

\item {\bf switches:}
    \begin{itemize}
      \item {\tt controlLog=<rootname>} --- Log input and output data to an SDDS file using the given rootname.
      \item {\tt interval=<real-value>} --- Specifies the interval in seconds between readings. The time
        interval is implemented with a call to usleep between calls to the control system.
        Because the calls to the control system may take up a significant amount of time, the average
        effective time interval may be longer than specified.
      \item {\tt steps=<integer-value>} --- Number of iterations for the feedforward before normal exiting.
      \item {\tt verbose} --- Prints out a message when data is taken.
      \item {\tt dryRun} --- Readback variables are read and computations are performed,
        but the control variables aren't changed.
      \item {\tt offsetOnly} --- Offset actuator tables using initial readback values so no changes are made until the readback changes.
      \item {\tt averageOf=<number>[,interval=<seconds>]} --- Specifies averaging of the
        readback values, giving the number of readings to average and the interval in seconds
        between the readbacks.
      \item {\tt advance=<seconds>} --- Advance the readback value by the given number of seconds based on the observed ramp.
      \item {\tt testValues=<file>} --- An SDDS file containing minimum and maximum values
        of PV's specifying a range outside of which the feedforward
        is temporarily suspended. The contents of the file are
        described above.
      \item {\tt ezcaTiming[=<timeout>,<retries>]} --- Sets EZCA timeout and retry parameters (obsolete).
      \item {\tt runControlPV=string=<string>[,pingTimeout=<value>]} --- Gives the name of a process variable to use for
        workstation-independent process control via the Run Control facility and the value of run control ping time out.
      \item {\tt runControlDescription=<string>} --- Gives a text description to write to
        the Run Control record for process identification.
      \item {\tt CASecurityTest} --- Specifies that before writing values to EPICS, a
        test should be done of the state of Channel Access security on the process variables,
        to see if the process is permitted to write values.
      \item {\tt pendIOtime=<seconds>} --- Maximum time to wait for Channel Access I/O operations.
      \item {\tt infiniteLoop} --- Run indefinitely instead of for a fixed number of steps.
    \end{itemize}

\item {\bf see also:}
    \begin{itemize}
% replacing <prog> with the program to be referenced:
    \item \progref{sddsexperiment}
    \end{itemize}
\item {\bf author: M. Borland, ANL}
\end{sddsprog}
