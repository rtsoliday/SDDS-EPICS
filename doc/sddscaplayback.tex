%
% Documentation for sddscaplayback
%
\begin{sddsprog}{sddscaplayback}
\item \textbf{description:}
\verb+sddscaplayback+ reads waveform data from a multi-page SDDS file and writes the values to
EPICS process variables at a fixed interval. Each page in the input file represents a set of
numerical values for a single PV, which may be sent either as absolute values or as differences
from the current value.

\item \textbf{examples:}
The following command plays back the waveforms in \verb+orbit.sdds+ once with a half-second
interval between pages.
\begin{verbatim}
sddscaplayback orbit.sdds -interval=0.5 -mode=absolute -repetitions=1
\end{verbatim}

\item \textbf{synopsis:}
\begin{verbatim}
usage: sddscaplayback <inputFile> -interval=<seconds> [-repetitions=<number>]\
-mode=\{absolute|differential\} [-triggerPV=<string>] [-controlNameParameter=<string>]\
[-dataColumn=<columnName>] [-offset=<integer>] [-timeOffset=<seconds>]\
[-factor=<value>] [-restore] [-verbose] [-pendIOtime=<seconds>]\
[-runControlPV=string=<string>,pingTimeout=<value>]\
[-runControlDescription=string=<string>]
\end{verbatim}

\item \textbf{files:}
The input file is an SDDS file containing a string parameter \verb+ControlName+ for the PV name
and a numerical column \verb+Value+ (or the column specified by \verb+-dataColumn+) for the
waveform data. Each page corresponds to a different PV.

\item \textbf{switches:}
\begin{itemize}
  \item {\tt -interval=<seconds>} --- Interval between sending pages.
  \item {\tt -repetitions=<number>} --- Number of times to loop through pages; default is infinite.
  \item {\tt -mode=\{absolute|differential\}} --- Send values directly or as differences from starting values.
  \item {\tt -triggerPV=<string>} --- PV that restarts playback when it changes.
  \item {\tt -controlNameParameter=<string>} --- Name of parameter holding PV names; defaults to {\tt ControlName}.
  \item {\tt -dataColumn=<columnName>} --- Column containing waveform data; defaults to {\tt Value}.
  \item {\tt -offset=<integer>} --- Offset into pages used as circular buffer.
  \item {\tt -timeOffset=<seconds>} --- Fine time offset less than the interval.
  \item {\tt -factor=<value>} --- Scale factor applied to waveform data.
  \item {\tt -restore} --- Restore PVs to starting values on exit.
  \item {\tt -verbose} --- Print status information.
  \item {\tt -pendIOtime=<seconds>} --- Channel access pend IO time.
  \item {\tt -runControlPV=string=<string>,pingTimeout=<value>} --- Enable run control monitoring.
  \item {\tt -runControlDescription=string=<string>} --- Description used with run control.
\end{itemize}

\item \textbf{see also:}
\begin{itemize}
  \item \progref{sddslogger}
  \item \progref{sddscaramp}
  \item \progref{sddssnapshot}
\end{itemize}

\item \textbf{author:} Michael Borland, ANL/APS.
\end{sddsprog}
