% squishPVs: minimize readback PVs by tweaking a single actuator.
\begin{sddsprog}{squishPVs}
\item \textbf{description:}
\verb+squishPVs+ minimizes the values of a set of readback process
variables (for example bpm readbacks) by varying one setpoint process
variable (for example a corrector magnet setpoint) which has
a physical influence on the readback process variables.
An input file defines one or more data sets of correction groups, each
consisting of a list of readback PVs and one actuator PV.
The method simulates the manual tweaking of physical devices that is often necessary
when the readbacks are very noisy or when obtaining a response matrix is not
worth the trouble.

The name \verb+squishPVs+ comes from the apparent squishing of the real-time
trajectory display when \verb+squishPVs+ is running.
\item \textbf{examples:}
%
The first-turn horizontal trajectory of the APS ring is minimized with this command:
\begin{verbatim}
squishPVs xTrajCorr.sdds
\end{verbatim}
where correction groups are defined in the file \verb+xTrajCorr.sdds+.
Part of the first correction group in the file \verb+xTrajCorr.sdds+
is shown below:
\begin{verbatim}
SDDS1
&description
 text = "Input file for first turn correction", &end
&parameter
 name = "CorrectorPV",  type = "string", &end
&parameter
 name = "Gain",  type = "double", &end
! LowerLimit and UpperLimit parameters are optional
&parameter
 name = "LowerLimit", type = "double", &end
&parameter
 name = "UpperLimit", type = "double", &end
&column
 name = "BpmPV",  type = "string", &end
! Offset PVs are optional
&column
 name = "OffsetPV", type = "string", &end
! Fixed offsets are optional and default to 0
&column
 name = "OffsetValue", type = "double" &end
! Weights are optional and default to 1
&column
 name = "Weight", type = "double", &end
&data
 mode = "ascii", &end
! table number 1
! CorrectorPV:
S1A:H1:CurrentAO
1.000000000000000e+00                   ! Gain
-150                                    ! LowerLimit
150                                     ! UpperLimit
360                                     ! number of rows
S1A:P1:ms:x S1A:P1:ms:x:SetpointAO 0.1 1
S1A:P2:ms:x S1A:P2:ms:x:SetpointAO 0.2 2
S1A:P3:ms:x S1A:P3:ms:x:SetpointAO 0.3 1
S1A:P4:ms:x S1A:P4:ms:x:SetpointAO 0.4 1
S1B:P5:ms:x S1B:P5:ms:x:SetpointAO 0.5 2
S1B:P4:ms:x S1B:P4:ms:x:SetpointAO 0.4 1
...
\end{verbatim}


\item \textbf{synopsis:}
\begin{verbatim}
usage: squishPVs <inputfile>
    [-count=<pvName>,<lower>,<upper>,<number>]
    [-averages=<number>,<seconds-between-readings>] [-stepSize=<value>]
    [-subdivisions=<number>,<factor>]
    [-upstep=count=<number>,factor=<value>]
    [-repeat={number=<integer> | forever}[,pause=<seconds>]]
    [-pendIOtime=<seconds>]
    [-settlingTime=<seconds>] [-verbose]
    [-criterion={mav|rms|min|max|average}] [-maximize]
    [-threshold=<value>] [-actionLevel=<value>]
    [-testValues=<file>,[limit=<number>]]
    [-runControlPV={string=<string>|parameter=<string>},pingTimeout=<value>]
    [-runControlDescription={string=<string>|parameter=<string>}]
\end{verbatim}

\item \textbf{files:}
\begin{itemize}
  \item \textbf{input file:}\par
The input file data pages define correction groups. A correction group
consists of a list of readback PVs and one corrector PV. The required column and parameters are:
  \begin{itemize}
    \item {\tt BpmPV} --- String column of readback PVs. The name \verb+BpmPV+ is due
                to the original application of \verb+squishPV+ where the first turn
                trajectory is reduced by tweaked by corrector magnets.
    \item {\tt OffsetPV} --- Optional string column of PVs to subtract from corresponding \verb+BpmPV+.
                This will have the effect of optimizing toward the values of the offsets for
                positive weights (the default).  The default offsets are all 0.
    \item {\tt Weight} --- Optional double-precision column of values by which to multiply the
                \verb+readback-offset+ values.  The default weight is 1.
    \item {\tt CorrectorPV} --- String parameter of the corrector PV used in the correction group.
    \item {\tt Gain} --- Double parameter for factor by which to multiply the
                value of the command line option \verb+stepSize+.
  \end{itemize}
\end{itemize}
\item \textbf{switches:}
\begin{itemize}
  \item {\tt -count=<pvName>,<lower>,<upper>,<number>} --- Optional. Specifies
                a condition after which a reading will be taken. The process variable
                \verb+<pvName>+ is read at 1 second intervals. If this
                PV is within the validity limits specified for this number of readbacks, then
                the PVs of the correction group are read.
  \item {\tt -averages=<number>,<seconds-between-readings>} --- Optional. Number of PV readings to average
                and the number of seconds to wait between readings.
  \item {\tt -stepSize=<value>} --- Optional. Initial step size to attempt for corrector PVs.
  \item {\tt -subdivisions=<number>,<factor>} --- Optionally specifies number and size of interval
                     subdivisions.
  \item {\tt -upstep=count=<number>,factor=<value>} --- Optionally specifies the number of steps
                to take in one direction before increasing the step size by the given factor.
  \item {\tt -repeat={number=<integer> | forever}[,pause=<seconds>]} --- Specifies repeated
                optimization, either a given number of times or indefinitely, with a pause in between.
                This is useful if a system needs periodic tuning up.
  \item {\tt -pendIOtime=<seconds>} --- Optionally specifies the channel-access pendIO timeout for reading PVs.
  \item {\tt -settlingTime=<seconds> } --- Optionally specifies the settling time after corrector PV changes.
  \item {\tt -criterion={mav|rms|min|max|average}} --- Optionally specifies the criterion for optimization.
                Mean-absolute-value is the default.
  \item {\tt -threshold=<value>} --- Specifies that the change in the criterion must be below the specified
        value in order to be considered a genuine improvement.
  \item {\tt -actionLevel=<value>} --- Specifies that the criterion must be above the given value
        in order for optimization to start.
  \item {\tt -maximize} --- Specifies that the criterion should be maximized.  The default is
                minimization.
  \item {\tt -verbose} --- Optionally requests output during run.
  \item {\tt -testValues=<file>,[limit=<number>]} --
                     file is a sdds format file containing minimum and maximum values
                     of PV's specifying a range outside of which the feedback
                     is temporarily suspended. Column names are ControlName,
                     MinimumValue, MaximumValue. Optional column names are
                     SleepTime, ResetTime.
                     limits is the maximum number of failure times. The program will be
                     terminated when the continuous failure times reaches the limit.
  \item {\tt -runControlPV={string=<string>|parameter=<string>},pingTimeout=<value>} ---
                    Specifies a run control PV by name or parameter with optional ping timeout.
  \item {\tt -runControlDescription={string=<string>|parameter=<string>}} ---
                    Specifies a run control description PV by name or parameter.
\end{itemize}

\item \textbf{see also:}
\begin{itemize}
% replacing <prog> with the program to be referenced:
  \item \progref{sddscontrollaw}
\end{itemize}
\item \textbf{author:} M. Borland, ANL
\end{sddsprog}
