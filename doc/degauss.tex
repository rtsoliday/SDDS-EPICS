% Template for making SDDS Toolkit manual entries.
\begin{latexonly}
\newpage
\end{latexonly}
\subsection{degauss}
\label{degauss}

\begin{itemize}
\item {\bf description:}
\verb+degauss+ reads an SDDS specification file listing magnets and degaussing parameters.
It configures the corresponding subroutine records and, unless otherwise requested, initiates
magnet degaussing cycles for each device.
\item {\bf examples:}
To configure magnets without starting degaussing and later query their state:
\begin{flushleft}{\tt
  degauss -configure booster.sdds\\
  degauss -query booster.sdds
}\end{flushleft}
\item {\bf synopsis:}
  \begin{flushleft}{\tt
    degauss [-q|-query|-c|-configure|-s|-stop]\\
    \phantom{degauss }[-p|-pend|-pendIoTime <seconds>] [--] <spec-file>
  }\end{flushleft}
\item {\bf switches:}
  \begin{itemize}
    \item {\tt -q\,|\,-query} --- Print degaussing state of devices in {\tt spec-file}.
    \item {\tt -c\,|\,-configure} --- Configure records for degaussing but do not start.
    \item {\tt -s\,|\,-stop} --- Halt degaussing of devices in {\tt spec-file}.
    \item {\tt -p\,|\,-pend\,|\,-pendIoTime <seconds>} --- Maximum time to wait for Channel Access I/O.
  \end{itemize}
\item {\bf author:} C. Saunders, J. Anderson, R. Soliday, H. Shang, ANL
\end{itemize}
