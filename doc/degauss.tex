% degauss: configure and trigger magnet degaussing cycles.
\begin{sddsprog}{degauss}
\item \textbf{description:}
\verb+degauss+ reads an SDDS specification file listing magnets and degaussing parameters.
It configures the corresponding subroutine records and, unless otherwise requested, initiates
magnet degaussing cycles for each device.
\item \textbf{examples:}
To configure magnets without starting degaussing and later query their state:
\begin{verbatim}
  degauss -configure booster.sdds
  degauss -query booster.sdds
\end{verbatim}
\item \textbf{synopsis:}
  \begin{verbatim}
    usage: degauss [-q|-query|-c|-configure|-s|-stop]
                   [-p|-pend|-pendIoTime <seconds>] [--] <spec-file>

  \end{verbatim}
\item \textbf{files:}
  The magnet specification \verb+spec-file+ is an SDDS file with one row per device.
  Required string columns \verb+ControlName+ and \verb+ControlType+ identify the target device or process variable.
  Optional \verb+SuffixName+ overrides the default subroutine record suffix.
  For standard degaussing each row also supplies \verb+DecayMinutes+, \verb+DecaySeconds+, \verb+PeriodSeconds+,
  \verb+MaxCurrent+, and \verb+NumDecay+.
  Tri-degauss extends this with \verb+LastPeakCurrent+ and \verb+OutputInterval+,
  while APSU tri-degauss uses \verb+FirstPeakCurrent+, \verb+NumCycles+, \verb+LastPeakCurrent+,
  \verb+OutputInterval+, and \verb+MaxStepSize+.
  \begin{verbatim}
  &description text="degaussing specification"&end
  &column name=ControlName, type=string &end
  &column name=ControlType, type=string &end
  &column name=DecayMinutes, type=long &end
  &column name=DecaySeconds, type=long &end
  &column name=PeriodSeconds, type=long &end
  &column name=MaxCurrent, type=double &end
  &column name=NumDecay, type=long &end
  &data mode=ascii, no_row_counts=1 &end
  Q1:Standardize,dev,0,30,10,5.0,3
  \end{verbatim}
\item \textbf{switches:}
\begin{itemize}
  \item {\tt -q\,|\,-query} --- Print degaussing state of devices in {\tt spec-file}.
  \item {\tt -c\,|\,-configure} --- Configure records for degaussing but do not start.
  \item {\tt -s\,|\,-stop} --- Halt degaussing of devices in {\tt spec-file}.
  \item {\tt -p\,|\,-pend\,|\,-pendIoTime <seconds>} --- Maximum time to wait for Channel Access I/O.
\end{itemize}
\item \textbf{author:} Claude Saunders, Janet Anderson, Robert Soliday, Hairong Shang, ANL/APS.
\end{sddsprog}
