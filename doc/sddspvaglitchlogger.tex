% Template for SDDS Toolkit manual entry.
\begin{sddsprog}{sddspvaglitchlogger}
\item {\bf description:}
\verb+sddspvaglitchlogger+ monitors EPICS PV Access process variables and
logs values whenever configured trigger or glitch conditions occur.  An input
SDDS file lists the PVs to read and optional trigger parameters; a separate
trigger file may be provided to define the triggers.  Data are written to
files in the given directory or using a supplied rootname.

\item {\bf example:}
The following command monitors PVs listed in \verb+pvlist.sdds+ for one day
with a one minute sample interval.
\begin{verbatim}
sddspvaglitchlogger pvlist.sdds logs -sampleInterval=60 -time=24,hours
\end{verbatim}

\item {\bf synopsis:}
\begin{verbatim}
usage: sddspvaglitchlogger <input> <outputDirectory>|<outputRootname>
       [-triggerFile=<filename>] [-lockFile=<filename>[,verbose]]
       [-sampleInterval=<secs>] [-time=<value>[,<units>]]
       [-circularBuffer=[before=<number>,][after=<number>]]
       [-delay=<steps>] [-holdoffTime=<seconds>] [-autoHoldoff]
       [-pendIOtime=<value>]
       [-inhibitPV=name=<name>[,pendIOTime=<seconds>][,waitTime=<seconds>]]
       [-conditions=<filename>,{allMustPass|oneMustPass}]
       [-verbose] [-watchInput]
       [-runControlPV=string=<string>,pingTimeout=<value>]
       [-runControlDescription=string=<string>]
\end{verbatim}

\item {\bf files:}
  \begin{itemize}
  \item {\bf input file:}\par
    SDDS file defining the PVs to log. Required columns include
    {\tt ControlName}, {\tt ReadbackName}, {\tt ReadbackUnits},
    {\tt Provider}, {\tt ExpectNumeric}, {\tt ExpectFieldType} and
    {\tt ExpectElements}. When no trigger file is supplied, parameters such as
    {\tt OutputRootname}, {\tt TriggerControlName}, and optional threshold and
    alarm values appear in the file.
  \item {\bf trigger file:}\par
    Optional SDDS file containing trigger definitions. Columns mirror the
    trigger-related parameters and may include {\tt GlitchScript} to specify a
    command executed when a trigger fires.
  \end{itemize}

\item {\bf switches:}
    \begin{itemize}
    \item {\tt -sampleInterval=<real>[,<units>]} --- interval between readings.
    \item {\tt -time=<real>[,<units>]} --- total logging time.
    \item {\tt -circularBuffer=[before=<n>,][after=<n>]} --- samples to keep around triggers.
    \item {\tt -delay=<steps>} --- delay reading after a trigger.
    \item {\tt -holdoffTime=<seconds>} --- wait time after a trigger before enabling new triggers.
    \item {\tt -autoHoldoff} --- set holdoff time based on post-trigger samples.
    \item {\tt -pendIOtime=<value>} --- maximum time to wait for PV replies.
    \item {\tt -inhibitPV=name=<name>[,pendIOTime=<seconds>][,waitTime=<seconds>]} --- inhibit logging when this PV is nonzero.
    \item {\tt -conditions=<file>,{allMustPass|oneMustPass}} --- require limits to be satisfied before logging.
    \item {\tt -triggerFile=<file>} --- read trigger definitions from a separate SDDS file.
    \item {\tt -lockFile=<file>[,verbose]} --- prevent multiple instances.
    \item {\tt -watchInput} --- reload input files if they change.
    \item {\tt -runControlPV=string=<name>,pingTimeout=<value>} --- run-control PV.
    \item {\tt -runControlDescription=string=<text>} --- run-control description record.
    \item {\tt -verbose} --- print messages while running.
    \end{itemize}

\item {\bf see also:}
    \begin{itemize}
    \item \progref{sddsglitchlogger}
    \item \progref{sddsmonitor}
    \end{itemize}

\item {\bf author:} H. Shang, ANL
\end{sddsprog}

