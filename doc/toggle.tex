% toggle: alternate between two sets of PV values.
\begin{sddsprog}{toggle}
\item \textbf{description:}
\verb+toggle+ alternates between two sets of process variable values stored in
snapshot files and/or
process variables specified on the command line.
Time intervals and the values of the process variables upon termination can be specified.
\item \textbf{examples:}
%
Two states of the APS storage ring injection magnets are restored for 10 seconds each by writing
values of process variables listed in the snapshot files SRin.snp1 and SRin.snp2:
\begin{verbatim}
toggle SRin.snp1 SRin.snp2 -interval=10,10
\end{verbatim}
In this case the magnets return to the initial state upon termination.

\item \textbf{synopsis:}
\begin{verbatim}
usage: toggle [snapshotfile1 [snapshotfile2]]
       [-controlName=PVname,value1[,value2] ...]
       [-interval=interval1[,interval2]] [-cycles=number]
       [-finalSet={original|first|second}]
       [-pendIOtime=seconds]
       [-prompt]
       [-singleShot[=noprompt|stdout][,resend]]
       [-verbose] [-warning]
\end{verbatim}

\item \textbf{files:}
\begin{itemize}
  \item \textbf{input files:}\par
The input files are valid snapshot files as described in \progref{sddscasr}. At least three
columns must be defined:
  \begin{itemize}
    \item {\tt ControlName} --- Required column. String column for the process variable or device name.
    \item {\tt ControlType} --- Required column. String column for the control name type. For a
                process variable name use ``pv''; for a device name use ``dev''.
    \item {\tt ValueString} --- String column containing the value to be restored as a character string.
  \end{itemize}
Optional columns are:
  \begin{itemize}
    \item {\tt RestoreMsg} --- Optional column. String column for the device set message if
                the {\tt ControlType} value is ``dev''.
                If this column is absent then the default read message is ``set''.
  \end{itemize}
        Both files must have the same set of PV names.
\end{itemize}
\item \textbf{switches:}
\begin{itemize}
%   \item {\tt -pipe[=input][,output]} --- The standard SDDS Toolkit pipe option.
  \item {\tt -controlName=<PVname>,<value1>[,<value2>]} --- Optional. Specifies a PV to be
                alternated. If only one value is given, then the second value
                is taken to be pre-existing. These values are synchronized with
                the snapshot files if they are present.
  \item {\tt -interval=<interval1>[,<interval2>]} ---  Optional. \verb+<interval1>+ and \verb+<interval2>+
                 are the durations of PV value sets 1 and 2, respectively.
                 If \verb+<interval2>+ isn't present, then \verb+<interval2>+=\verb+<interval1>+.
                 If this option isn't present, then the default interval is 1 second each.
  \item {\tt -cycles=<number>} --- Optional number of cycles. Default is 1.
  \item {\tt -finalSet=\{original|first|second\}} --- Optional. Specifies which PV value set to apply at normal termination.
                 \verb+original+, \verb+first+, and \verb+second+ mean the pre-existing values, first set, and second set respectively.
                 During abnormal termination, the PVs are returned to their pre-existing values.
  \item {\tt -pendIOtime=<seconds>} --- Optional time to wait for channel access operations.
  \item {\tt -prompt} --- Optionally toggles values only on a \verb+<CR>+ key press.
  \item {\tt -singleShot[=noprompt|stdout][,resend]} --- Optionally performs a single toggle initiated by a \verb+<CR>+ key press. Qualifiers control prompting and resending the value.
  \item {\tt -verbose} --- Optionally prints extra information.
  \item {\tt -warning} --- Optionally prints warning messages.
\end{itemize}

\item \textbf{see also:}
\begin{itemize}
% replacing <prog> with the program to be referenced:
  \item \progref{sddscasr}
\end{itemize}
\item \textbf{author:} L. Emery, ANL
\end{sddsprog}
