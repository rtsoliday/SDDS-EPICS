% sddsoptimize: optimize readback RMS by varying setpoints.
\begin{sddsprog}{sddsoptimize}
\item \textbf{description:}
\verb+sddsoptimize+ optimizes the RMS of a set of readback process variables by automatically varying setpoint process variables (or knobs composed of setpoint PVs), which have a physical influence on the readback process variables, through simplex or 1dscan method.

\item \textbf{examples:}
The trajectory of the booster BPMs is controlled with this command:
\begin{verbatim}
sddsoptimize -measFile=booster.h.moni -varFile=vv -simplex=evaluations=50,divisions=12\\
\quad -knobFiles=booster.cokn -verbose
\end{verbatim}
where the contents of the file \verb+booster.h.moni+ are
\begin{verbatim}
SDDS1
&column
 name = "ControlName",  type = "string", &end
&column
 name = "ReadbackName",  type = "string", &end
&column
 name = "ReadbackUnits",  type = "string", &end
&data
 mode = ascii, no_row_counts=1 &end
oag:B1C0P1:ms:x B1C0P1:ms:x mm
oag:B1C0P2:ms:x B1C0P2:ms:x mm
oag:B1C1P1:ms:x B1C1P1:ms:x mm
oag:B1C1P2:ms:x B1C1P2:ms:x mm
.......

\end{verbatim}
the contents of the file \verb+vv+ are
\begin{verbatim}
SDDS1
&parameter name=PauseAfterChange, type=double, &end
&column name=ControlName, type=string,  &end
&column name=LowerLimit, type=double,  &end
&column name=UpperLimit, type=double,  &end
&column name=InitialChange, type=double,  &end
&data mode=ascii,no_row_counts=1,&end
oag:B1C1H:KickAO -2  2  2
oag:B1C2H:KickAO -2.500000000000000e+01  2.500000000000000e+01  2.000000000000000e-01
oag:B1C3H:KickAO -2.500000000000000e+01  2.500000000000000e+01  2.000000000000000e-01
B:h11cos         -25                     25                     0.1
B:h12sin         -25                     25                     0.1
B:h12cos         -25                     25                     0.1
.....

\end{verbatim}
the contents of the file \verb+booster.cokn+ are
\begin{verbatim}
SDDS1
&parameter name=ControlName, type=string, &end
&parameter name=KnobDescription, type=string, &end
&parameter name=Gain, type=double, &end
&parameter name=ControlType, type=string, &end
&parameter name=ControlUnits, type=string, &end
&parameter name=Filename, description="Name of file from which this page came", type=string, &end
&parameter name=NumberCombined, description="Number of files combined to make this file", type=long, &end
&column name=ControlName, type=string,  &end
&column name=Weight, type=double,  &end
&data mode=asc_ii,no_row_counts=1. &end
....
....
!.page 2
B:h11cos
h plane 11th harm. cos (1 mrad/click)
1.0
pv
rad
11HarmCosineh.cokn
8
oag:B1C0H:KickAO 0.99997256549446789
oag:B1C1H:KickAO -0.21202791415796657
oag:B1C2H:KickAO -0.93188756664225914
.....
.....

\end{verbatim}

\item \textbf{synopsis:}
\begin{verbatim}
usage: sddsoptimize [-measFile=<file>|-measScript=<script>] [-varScript=<script>]\
[-restartFile=<file>] [-pendIOtime=<seconds>] [-conditioning=<script>]\
-varFile=<file> -knobFiles=<file1>[,<file2>...]\
[-simplex=[restarts=<nRestarts>][,cycles=<nCycles>][,evaluations=<nEvals>][,no1dscans][,divisions=<int>][,randomSigns]]\
[-rcds=[cycles=<number>][,evaluations=<number>][,step=<value>][,noise=<value>][,dmatFile=<file>][,useMinForBracket][,verbose]]\
[-1dscan=[divisions=<value>][,cycles=<number>][,evaluations=<value>][,refresh]]\
[-logFile=<file>] [-extraLogFile=<file>] [-dryRun] [-verbose] [-tolerance=<value>] [-maximize]\
[-target=<value>] [-testValues=file=<file>[,limit=<count>]]\
[-runControlPV={string=<string>|parameter=<string>},pingTimeout=<value>,pingInterval=<value>]\
[-runControlDescription={string=<string>|parameter=<string>}]
\end{verbatim}
Perform optimization on APS control system process variables using simplex, 1dscan, or RCDS methods.
\item \textbf{files:}
\begin{itemize}
  \item \textbf{variable input file:} \par
The variable input file is an SDDS file with one string column: ControlName, which is required
and gives the list of control correctors (process variables or knobs). It also contains three
double columns: LowerLimit, UpperLimit, IntialChange and InitialValue. InitialValue column
is optional. Others are required. InitialChange column specifies the initial changes to the correctors.
Variable input file has one parameter --PauseBetweenReadings(double), which sets the waiting time in
seconds between two settings of the correctors.
  \item \textbf{measurement file:} \par This file specifies the measurement to be optimized.
It has four columns:
  \begin{itemize}
    \item {\tt ControlName} --- Required string column. Gives the list of process variables
                 to be controlled.
    \item {\tt ReadbackName} --- string, optional.
    \item {\tt ReadbackUnits} --- string, optional.
    \item {\tt Weight} -- double, optional. Defines the weight of each PV contributed to RMS.
  \end{itemize}
It has three parameters:
  \begin{itemize}
    \item {\tt Tolerance} --- double, sets the converging limit.
    \item {\tt NumberToAverage} --- long, sets number of average for measurement PVs.
    \item {\tt PauseBetweenReadings} --- double, sets interval between two readings.

  \end{itemize}

  \item \textbf{knob file:} \par
To make \verb+sddsoptimize+ more robust, one can implement optimizaion on knobs,
that are composed of set point process variables. The process variables that a knob contains are
given in knob file, which contain following parameters and columns:
  \begin{itemize}
    \item {\tt ControlName} --- Required string parameter. The name of knob, which
                acts as a corrector as other PVs do.
    \item {\tt ControlName} --- Required string column. The names of PVs the knob
                specified above contains.
    \item {\tt Weight} --- Required double columns. Defines the weights of PVs that
                compose the knob.
    \item {\tt Gain} ---Optional double parameter. The value of each PV the knob contains
                is value(PV)=value(knob)*Gain*Weight(PV). When it is not given, set it to 1.
    \item {\tt ControlType} --- Optional string parameter. Specifies the control type.
    \item {\tt ControlUnits}  --- Optional string parameter.
                 Specifies the units of control PVs.
    \item {\tt KnobDescription} -- Optional string parameter.
    \item {\tt Filename}--- Optional string parameter. The name of the file where
                this knob comes from.
    \item {\tt Numbercombined} --- Optional long parameter. The number of files combined.
                The resulted knob file contains all the information of the files combined.
                The content of each page is from the combined file specified in filename parameter.
  \end{itemize}

  \item \textbf{output log file:} \par
The output file contains one data column for each process variables defined in the variable
input file. By default, the data type is double. One row is written at every evaluation.
Also two more columns and two parameters are defined:
  \begin{itemize}
    \item {\tt EvalIndex} --- Long Column. The index of evaluations.
    \item {\tt currentValue} --- Double column. The RMS value of measurement at each evaluaion.
    \item {\tt variableFile} --- String parameter. The name of input variable file.
    \item {\tt measurementFile} --- String parameter. The name of input measurement file.
  \end{itemize}

\end{itemize}

\item \textbf{switches:}
\begin{itemize}
%   \item {\tt -pipe[=input][,output]} --- The standard SDDS Toolkit pipe option.
  \item {\tt -varFile=<file>} --- Required; defines control correctors or knobs.
  \item {\tt -measFile=<file>} --- Specifies the measurement file.
  \item {\tt -knobFiles=<file1>[,<file2>...]} --- List of knob definition files.
  \item {\tt -measScript=<script>} --- Script for measuring PVs.
  \item {\tt -varScript=<script>} --- Script for setting setpoint PVs or tags.
  \item {\tt -restartFile=<file>} --- Save optimized values for later use.
  \item {\tt -logFile=<file>} --- Write optimization data to an SDDS log.
  \item {\tt -extraLogFile=<file>} --- Additional PVs for logging.
  \item {\tt -pendIOtime=<seconds>} --- Channel access pendIO timeout.
  \item {\tt -conditioning=<script>} --- Conditioning script executed before optimization.
  \item {\tt -simplex} --- Parameters for simplex optimization.
  \item {\tt -rcds} --- Parameters for robust conjugate direction search.
  \item {\tt -1dscan} --- Parameters for one-dimensional scan optimization.
  \item {\tt -dryRun} --- Compute settings without applying them.
  \item {\tt -verbose} --- Print additional progress information.
  \item {\tt -target=<value>} --- Target value for RMS of measurement PVs.
  \item {\tt -tolerance=<value>} --- Tolerance when using a measurement script.
  \item {\tt -maximize} --- Maximize instead of minimize the measurement.
  \item {\tt -runControlPV} --- Specifies the runControl PV record.
  \item {\tt -runControlDescription} --- Specifies a runControl PV description record.
  \item {\tt -testValues=file=<file>[,limit=<count>]} --- Enforce temporary feedback limits.
\end{itemize}
\item \textbf{see also:}
\begin{itemize}
  \item \progref{sddsexperiment}
\end{itemize}
\item \textbf{author:} Hairong Shang, ANL/APS.
\end{sddsprog}
