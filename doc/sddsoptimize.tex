%
% Template for making SDDS Toolkit manual entries.
%
\begin{latexonly}
\newpage
\end{latexonly}

%
% Substitute the program name for sddsmonitor
%
\subsection{sddsoptimize}
\label{sddsoptimize}

\begin{itemize}
\item {\bf description:}
%
% Insert text of description (typically a paragraph) here.
%
\verb+sddsoptimize+ optimizes the RMS of a set of readback process variables by automatically varying setpoint process variables (or knobs composed of setpoint PVs), which have a physical influence on the readback process variables, through simplex or 1dscan method.   

\item {\bf example:} 
%
% Insert text of examples in this section.  Examples should be simple and
% should be proceeded by a brief description.  Wrap the commands for each
% example in the following construct:
% 
%
The trajectory of the booster BPMs is controlled with this command:
\begin{verbatim}
sddsoptimize -measFile= booster.h.moni -varFile=vv -simplex=evaluations=50,divisions=12 \
             -knobFiles=booster.cokn -verbose
\end{verbatim}
where the contents of the file \verb+booster.h.moni+ are
\begin{verbatim}
SDDS1
&column
 name = "ControlName",  type = "string", &end
&column
 name = "ReadbackName",  type = "string", &end
&column             
 name = "ReadbackUnits",  type = "string", &end
&data
 mode = ascii, no_row_counts=1 &end
oag:B1C0P1:ms:x B1C0P1:ms:x mm
oag:B1C0P2:ms:x B1C0P2:ms:x mm
oag:B1C1P1:ms:x B1C1P1:ms:x mm
oag:B1C1P2:ms:x B1C1P2:ms:x mm
.......

\end{verbatim}
the contents of the file \verb+vv+ are
\begin{verbatim}
SDDS1
&parameter name=PauseAfterChange, type=double, &end
&column name=ControlName, type=string,  &end
&column name=LowerLimit, type=double,  &end
&column name=UpperLimit, type=double,  &end
&column name=InitialChange, type=double,  &end
&data mode=ascii,no_row_counts=1,&end
oag:B1C1H:KickAO -2  2  2 
oag:B1C2H:KickAO -2.500000000000000e+01  2.500000000000000e+01  2.000000000000000e-01 
oag:B1C3H:KickAO -2.500000000000000e+01  2.500000000000000e+01  2.000000000000000e-01 
B:h11cos         -25                     25                     0.1
B:h12sin         -25                     25                     0.1
B:h12cos         -25                     25                     0.1
.....

\end{verbatim}
the contents of the file \verb+booster.cokn+ are
\begin{verbatim}
SDDS1
&parameter name=ControlName, type=string, &end
&parameter name=KnobDescription, type=string, &end
&parameter name=Gain, type=double, &end
&parameter name=ControlType, type=string, &end
&parameter name=ControlUnits, type=string, &end
&parameter name=Filename, description="Name of file from which this page came", type=string, &end
&parameter name=NumberCombined, description="Number of files combined to make this file", type=long, &end
&column name=ControlName, type=string,  &end
&column name=Weight, type=double,  &end
&data mode=asc_ii,no_row_counts=1. &end
....
....
!.page 2
B:h11cos
h plane 11th harm. cos (1 mrad/click)
1.0
pv
rad
11HarmCosineh.cokn
8
oag:B1C0H:KickAO 0.99997256549446789
oag:B1C1H:KickAO -0.21202791415796657
oag:B1C2H:KickAO -0.93188756664225914
.....
.....

\end{verbatim}

\item {\bf synopsis:} 
%
% Insert usage message here:
%
\begin{verbatim}
usage: sddsoptimize -measFile=<filename> -measScript=<script> 
        [-varScript=<scriptname>]
       -varFile=<filename> -knobFiles=<filename1> , <filename2>,... 
       [-simplex=[restarts=<nRestarts>][,cycles=<nCycles>,] 
       [evaluations=<nEvals>,][no1dscans][,divisions=<int>]] 
       [-logFile=<filename>] [-verbose] [-tolerance=<value>] [-maximize] 
       [-1dscan=[divisions=<value>,][cycles=<number>][,evaluations=<value>][,refresh]]
       [-target=<value>] [-testValues=file=<filename>[,limit=<count>]]
       [-runControlPV={string=<string>|parameter=<string>},pingTimeout=<value>,
         pingInterval=<value>
Perform optimization on APS control system process variables using simplex or 1dscan method.
\end{verbatim}
\item {\bf files:}
% Describe the files that are used and produced
\begin{itemize}
\item {\bf variable input file:} \par
The variable input file is an SDDS file with one string column: ControlName, which is required
and gives the list of control correctors (process variables or knobs). It also contains three
double columns: LowerLimit, UpperLimit, IntialChange and InitialValue. InitialValue column
is optional. Others are required. InitialChange column specifies the initial changes to the correctors.
Variable input file has one parameter --PauseBetweenReadings(double), which sets the waiting time in
seconds between two settings of the correctors.
\item {\bf measurement file:} \par This file specifies the measurement to be optimized. 
It has four columns:
\begin{itemize}
        \item {\tt ControlName} --- Required string column. Gives the list of process variables
                 to be controlled.
        \item {\tt ReadbackName} --- string, optional.
        \item {\tt ReadbackUnits} --- string, optional.
        \item {\tt Weight} -- double, optional. Defines the weight of each PV contributed to RMS.
\end{itemize}
It has three parameters: 
\begin{itemize}
        \item {\tt Tolerance} --- double, sets the converging limit.
        \item {\tt NumberToAverage} --- long, sets number of average for measurement PVs.
        \item {\tt PauseBetweenReadings} --- double, sets interval between two readings.

\end{itemize}

\item {\bf knob file:} \par
To make \verb+sddsoptimize+ more robust, one can implement optimizaion on knobs,
that are composed of set point process variables. The process variables that a knob contains are
given in knob file, which contain following parameters and columns:
\begin{itemize}
        \item {\tt ControlName} --- Required string parameter. The name of knob, which
                acts as a corrector as other PVs do.
        \item {\tt ControlName} --- Required string column. The names of PVs the knob
                specified above contains.
        \item {\tt Weight} --- Required double columns. Defines the weights of PVs that
                compose the knob.
        \item {\tt Gain} ---Optional double parameter. The value of each PV the knob contains
                is value(PV)=value(knob)*Gain*Weight(PV). When it is not given, set it to 1.
        \item {\tt ControlType} --- Optional string parameter. Specifies the control type.
        \item {\tt ControlUnits}  --- Optional string parameter.
                 Specifies the units of control PVs.
        \item {\tt KnobDescription} -- Optional string parameter.
        \item {\tt Filename}--- Optional string parameter. The name of the file where 
                this knob comes from.
        \item {\tt Numbercombined} --- Optional long parameter. The number of files combined.
                The resulted knob file contains all the information of the files combined.
                The content of each page is from the combined file specified in filename parameter.
\end{itemize}

\item {\bf output log file:} \par
The output file contains one data column for each process variables defined in the variable 
input file. By default, the data type is double. One row is written at every evaluation.
Also two more columns and two parameters are defined: 
\begin{itemize}
        \item {\tt EvalIndex} --- Long Column. The index of evaluations.
        \item {\tt currentValue} --- Double column. The RMS value of measurement at each evaluaion.
        \item {\tt variableFile} --- String parameter. The name of input variable file.
        \item {\tt measurementFile} --- String parameter. The name of input measurement file.
\end{itemize}

\end{itemize}

%
\item {\bf switches:}
%
% Describe the switches that are available
%
    \begin{itemize}
%
%   \item {\tt -pipe[=input][,output]} --- The standard SDDS Toolkit pipe option.
%
        \item {\tt -varFile} --- required, see above. Data in \verb+ControlName+may be valid process
                variable names or knobs defined in knob files.
        \item {\tt -measFile} ---specifies the name of measurement file. See above about its content.
         \item {\tt -knobFiles} ---specifies list of knob files, which contain the knob correctors
                given in varFile. See above about its content.
        \item {\tt -measScript=<measScript>} --- user given script for measuring PVs. Either -measScript or -measFile
                is given for measurement. {\\tt<measScript>} is an executable script which is called by optimizer and
                outputs a value to the stand outputs. This value is the evaluation value and read by the optimizer.
        \item {\tt -varScript=<varScript>} --- user given script for setting setpoint PVs or tags. 
                If both -measScript and -varScript are given, there are no obvious ioc calls.
                The calling syntax of {\tt<varScript>} is: {\tt<varScript> -tagList <tagList> -valueList <valueList>}
                where {\tt<varScript>} is an executable command, {\tt<tagList>} is a list of setpoints or 
                tag names supplied by the varFile and {\tt<valueList>} is a list of values for 
                setting the setpoints or tags. These values are calculated by optimizer in each evaluation.
        \item {\tt -simplex} ---  Give parameters of the simplex optimization. Each start or restart 
                allows <nCycles> cycles with up to <nEvals> evaluations of the function.  Defaults are
                1 restarts, 1 cycles, and 100 evaluations (iterations). If no1dscan is given, then turn
                off the 1D scan function in simplex. Divisions gives the parameter of maximum divisions
                used in simplex. If repeat is given, then read the previous better point again to keep
                track with the noise. If there is not much noise, it should be turned off to have a 
                better performance.
        \item {\tt -1dscan} --- Give parameters of one dimensional scan optimization. Cycles gives the max.
            number of cycles (i.e. loops) of 1dscan and with an maximum number of 
            evaluations. Divisions is the max. number of division applied in 1dscan. 
            Either 1dscan or simplex method is used for optimize. 
            repeat: read the measurement again if the variable PVs are set back
            to their previous values to keep track with noise.
        \item {\tt -verbose } --- Specifies printing of possibly useful data to the standard output.
        \item {\tt -target} --- the target value for RMS of measurement PV (or PVs).
        \item {\tt -tolerance} --- tolerance should be given if measScript is used, otherwise,
            a default value of 0.001 is set to tolerance.
        \item {\tt -maximize} --- If maximize option is given, sddsoptimize maximizes measurement by
                varying control correctors. Otherwise, it minimizes the measurement.
        \item {\tt -runControlPV} -- specifies the runControl PV record.
        \item {\tt -runControlDescription} -- specifies a string parameter whose value is a 
                runControl PV description record.
        \item {\tt -testValues} -- <filename> is name of an sdds format file containing minimum 
            and maximum values of PV's specifying a range outside of which the feedback is temporarily
            suspended. Column names are ControlName, MinimumValue, MaximumValue. limit specifies
            the maximum times of testing when test fails.
  \end{itemize}
\item {\bf see also:}
    \begin{itemize}
%
% Insert references to other programs by duplicating this line and 
% replacing <prog> with the program to be referenced:
%
    \item \progref{sddsexperiment}
    \end{itemize}
%
% Insert your name and affiliation after the '}'
%
\item {\bf author: H. Shang, ANL}
\end{itemize}
